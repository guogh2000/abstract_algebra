\section{CHAPTER2习题}
\begin{problem}[P97T1]
    已知$G$是有限群,$N\lhd G$,$\l(|N|,[G:N]\r)=1$.证明:若元素$a$的阶整除$|N|$,则$a\in N$.
\end{problem}
\begin{proof}
    考虑自然同态\begin{align*}
        \pi:G\to&G/N\\
        g\mapsto&gN,
    \end{align*}于是\begin{align*}
        \l[\pi(a)\r]^{o(a)}=&(aN)^{o(a)}\\
        \xlongequal[]{N\lhd G}&a^{o(a)}N^{o(a)}\\
        =&N,
    \end{align*}又$N$是$G/N$的幺元,所以\begin{align}\label{snjdkk}
        \l|\pi(a)\r|\Big|\l|o(a)\r|\Big|\l|N\r|.
    \end{align}

    注意到\begin{align*}
        aN\in G/N,
    \end{align*}所以\begin{align*}
        \l|aN\r|\Big|\l[G:N\r],
    \end{align*}即\begin{align}\label{asjgndk}
        \l|\pi(a)\r|\Big|\l[G:N\r].
    \end{align}

    由式\eqref{snjdkk}与\eqref{asjgndk}结合$\l(|N|,[G:N]\r)=1$可知\begin{align*}
        &|\pi(a)|=1\\
        \Lra&\pi(a)=N\\
        \Lra&aN=N\\
        \Lra&a\in N,
    \end{align*}问题得证.
\end{proof}
\begin{problem}[P97T2]
    设$c$是群$G$中阶为$rs$的元素,其中$(r,s)=1$.证明$c$可以表示成$c=ab$,其中$a$的阶为$r,b$的阶为$s$,且$a,b$都是$c$的方幂.
\end{problem}
\begin{proof}
    由$(r,s)=1$可知\begin{align*}
        \exists u,v\in\Z\st ur+vs=1,
    \end{align*}不难验证\begin{align*}
        &a=c^{vs}\\
        &b=c^{ur}
    \end{align*}满足题设要求.
\end{proof}
\begin{problem}[P97T3]
    已知群$G$中元素$a$的阶与正整数$k$互素,证明方程$x^k=a$在$<a>$内恰有一解.
\end{problem}
\begin{proof}
    由$\l(o(a),k\r)=1$可知\begin{align*}
        \exists u,v\in \Z\st uo(a)+vk=1,
    \end{align*}所以\begin{align*}
        a^{uo(a)+vk}=a,
    \end{align*}即\begin{align*}
        a^{vk}=a,
    \end{align*}由此可见$x=a^v$是$x^k=a$在$<a>$内的解.

    现设$x=a^v_0$也是$x^k=a$在$<a>$内的解,其中$0\leq v-v_0\leq o(a)-1$,于是\begin{align*}
        &a^{vk}=a\\
        &a^{v_0k}=a,
    \end{align*}即\begin{align*}
        vk\equiv1&\mod o(a)\\
        v_0k\equiv1&\mod o(a),
    \end{align*}所以\begin{align*}
    o(a)|(v-v_0)k,
    \end{align*}由此可见$v-v_0=0$即$v=v_0$.

    综上,问题证毕.
\end{proof}
\begin{problem}[P97T4]
    证明在群中,$ab$与$ba$有相同的阶.
\end{problem}
\begin{proof}
    注意到$ab=b^{-1}\cdot ba\cdot b$,所以\begin{align*}
        &ab=e\\
        \iff&b^{-1}\cdot ba\cdot b=e\\
        \iff&ba=e,
    \end{align*}可见问题得证.
\end{proof}
\begin{problem}[P97T10]
    证明$S_n$中的任意一个置换能由$n-1$个对换$(12),(13),\cdots,(1n)$生成,也能由$n-1$个对换$(12),(23),\cdots,(n-1\ n)$生成.
\end{problem}
\begin{proof}
    由引理\ref{p65yl}可知$S_n$中的任意一个置换都可以拆成若干个对换$(ij)$的复合,注意到\begin{align*}
        (1i)(1j)(1i)=(ij),
    \end{align*}所以任意一个置换都能由$(n-1)$个对换\begin{align*}
        (12),(13),\cdots,(1n)
    \end{align*}生成,第一个论断证毕.

    注意到\begin{align*}
        (1i)(i\ i+1)(1i)=(1\ i+1),
    \end{align*}故可用归纳法证明\begin{align*}
        (12),(23),\cdots,(n-1\ n)
    \end{align*}能生成\begin{align*}
        (12)(13),\cdots,(1n),
    \end{align*}再结合第一个论断可知它能生成$S_n$中的任意一个置换,第二个论断证毕.
\end{proof}
\begin{problem}[P97T16]
    设$H_1,H_2$是群$G$的两个子群,证明$H_1\cap H_2$的任一左陪集是$H_1$的一个左陪集与$H_2$的一个左陪集的交.
\end{problem}
\begin{proof}
    若能证明对$\forall g\in G$都有$g(H_1\cap H_2)=gH_1\cap gH_2$,则问题得证.证明思路为两者互相包含.

    (1)对$\forall$给定的$g\in G$,在$g(H_1\cap H_2)$中任取$gh_0$,有\begin{align*}
        &h_0\in H_1\cap H_2\\
        \Lra&gh_0\in gH_1,gh_0\in gH_2\\
        \Lra&gh_0\in gH_1\cap gH_2\\
        \Lra&g(H_1\cap H_2)\subset gH_1\cap gH_2.
    \end{align*}

    (2)反之,对任意给定的$g\in G$,在$gH_1\cap gH_2$中任取$gh_0$,有\begin{align*}
        &gh_0\in gH_1,gh_0\in gH_2\\
        \xLra[]{g\text{的任意性}}&h_0\in H_1,h_0\in H_2\\
        \Lra&h_0\in H_1\cap H_2\\
        \Lra&gh_0\in g(H_1\cap H_2)\\
        \Lra&gH_1\cap gH_2\subset g(H_1\cap H_2).
    \end{align*}

    由(1),(2)便知$g(H_1\cap H_2)=gH_1\cap gH_2$,综上,问题得证.
\end{proof}
\begin{problem}[P98T18]\label{p98t18}
    设$G$为有限群,$H<G$且$[G:H]=n>1$.证明$G$或者含有指数能整除$n!$的非平凡正规子群,或者$G$同构于$S_n$的一个子群.
\end{problem}
\begin{proof}
    设群$G$在子群$H$上的传递置换表示(定义\ref{idvhbu})为\begin{align*}
        \phi:G\to S_n.
    \end{align*}
    
    由$[G:H]>1$可知\begin{align}
        &\im\phi>1\nonumber\\
        \Lra&\im\phi\neq\l\{e\r\}\nonumber\\
        \Lra&\ker\phi\neq G\label{jnakd}.
    \end{align}

    (1)若$\ker\phi\neq\l\{e\r\}$,结合\eqref{jnakd}可知\begin{align}
        \text{$\ker\phi$是$G$的非平凡正规子群.}\label{uhnj}
    \end{align}注意到\begin{align}
        |\ker\phi|\Big|[G:H]\xlongequal[]{\text{群同态基本定理(定理\ref{qttjbdy})}}|\im\phi|\Big||S_n|=n!,\label{sjda}
    \end{align}所以由\eqref{uhnj}与\eqref{sjda}可知$\ker\phi$是$G$的指数能够整除$n!$的非平凡的正规子群.
    
    (2)若$\ker\phi=\l\{e\r\}$,则\begin{align*}
        &G/\ker\phi\cong\im\phi\\
        \iff&G\cong\im\phi<S_n,
    \end{align*}即$G$同构于$S_n$的子群$\im\phi$.

    由(1),(2)便知问题证毕.
\end{proof}
\begin{problem}[P98T19]
    设$G$为有限群,$p$是$|G|$的最小素因子.证明指数(定义\ref{vs})为$p$的子群(若存在)必正规.
\end{problem}
\begin{proof}
    设$H$是群$G$的指数为$p$的子群,即$[G:H]=p$,$\phi$是$G$在$H$上的传递置换表示(定义\ref{idvhbu}).

    (1)若$\ker\phi=\l\{e\r\}$,则由群同态基本定理(定理\ref{qttjbdy})可知\begin{align*}
        &G/\ker\phi\cong\im\phi\\
        \iff&G\cong\im\phi<S_p,
    \end{align*}于是\begin{align*}
        |G|=|\im\phi|\Big||S_p|=p!,
    \end{align*}结合$p$是$|G|$的最小素因子便知$|G|=p$,于是\begin{align*}
        &|H|=\frac{|G|}{[G:H]}=\frac{p}{p}=1\\
        \iff&H=\l\{e\r\}\\
        \Lra\ &H\text{是$G$的正规子群}.
    \end{align*}

    (2)若$\ker\phi\neq\l\{e\r\}$,则\begin{align*}
        &g\in\ker\phi\\
        \iff&\forall a_i\in G,ga_iH=a_iH\\
        \xLra[]{\text{当$a_i=e$时}}\ &gH=H\\
        \iff&g\in H,
    \end{align*}由此可见$\ker\phi\subset H$,从而$\ker\phi\lhd H$.注意到\begin{align*}
        &[G:\ker\phi]=|\im\phi|\Big||S_p|=p!\\
        \xLra[]{\text{$p$是$|G|$的最小素因子}}&[G:\ker\phi]=p\\
        \xLra[]{\ker\phi\lhd H,[G:H]=p}&\text{$H=\ker\phi$是$G$的指数为$p$的正规子群}.
    \end{align*}

    由(1),(2)可知问题证毕.
\end{proof}
\begin{problem}[P98T21]
    证明任一非交换的$6$阶群同构于$S_3$.
\end{problem}
证明该问题需要用到以下引理.
\begin{lemma}\label{sdfmk}
    素数阶群都是循环群.
\end{lemma}
\begin{proof}
    任取阶为素数$p$的有限群$G,e\neq g\in G$,于是由Lagrange定理(推论\ref{Lagrangedl})可知\begin{align*}
        &\exists k\in\N_+\st|G|=k|<g>|\\
        \xLra[]{\text{$p$是素数,$g\neq e$}}&|G|=|<g>|\\
        \Lra&G=<g>,
    \end{align*}亦即$G$是循环群,由$G$的任意性可知问题得证.
\end{proof}
\begin{proof}
    设$G$是非交换的$6$阶群.注意到\begin{align*}
        6=2\times3,
    \end{align*}所以由$Sl$第一定理(定理\ref{xldydl})可知$G$有阶为$2$的\Sl2$A$,阶为$3$的\Sl3$B$,由引理\ref{sdfmk}可知$A,B$都是循环群,故可设\begin{align*}
        \text{$A=<\alpha>,B=<\beta>$,其中$o(a)=2,o(b)=3$.}
    \end{align*}
    由于$G$不交换,故可用反证法证明\begin{align*}
        &\text{$\alpha,\beta$不交换}\\
        \Lra&\alpha\neq\beta,\alpha\neq\beta^2\\
        \xLra[]{\beta\cdot\beta^2=\beta^3=e}&\alpha\neq\beta^{-1}\\
        \Lra&\alpha\notin B,
    \end{align*}所以$B$的左陪集$B,\alpha B$满足\begin{align*}
        &B\cap\alpha B=\emptyset,|B|+|\alpha B|=|B|+|B|=3+3=6=|G|\\
        \Lra&G=B\cup\alpha B=\l\{e,\beta,\beta^2,\alpha,\alpha\beta,\alpha\beta^2\r\}.
    \end{align*}定义\begin{align*}
        \phi:G\to&S_6\\
        \alpha\mapsto&(12)\\
        \beta\mapsto&(123),
    \end{align*}利用上述讨论不难验证这是良定义的同构映射,从而\begin{align*}
        G\cong S_6.
    \end{align*}
\end{proof}
\begin{problem}[P98T29]
    已知$A,B$是有限群$G$的两个非空子集.若$|A|+|B|>|G|$,则$AB=G$.
\end{problem}
\begin{proof}
    反设$AB\neq G$,由于$AB\subset G$,所以\begin{align*}
        &G/AB\supset\l\{AB\r\}\\
        \Lra&[G:AB]>1\\
        \Lra&\exists g\in G\st g\notin AB\\
        \Lra&\exists g\in G\st\l\{g\r\}\cap AB=\emptyset\\
        \Lra&\exists g\in G\st gB^{-1}\cap A=\emptyset\\
        \xLra[]{|gB^{-1}|=|B^{-1}|=|B|}&\text{$G$中至少有$|B|$个元素不在$A$中}\\
        \xLra[]{|G|\neq\infty}&|G|-|A|\geq|B|,
    \end{align*}矛盾,假设不成立,从而$AB=G$.
\end{proof}
\begin{problem}[P98T33]
    已知$G$是有限群,$N\lhd G,P$为$N$的\Sl{p}.证明$G=N\cdot N_G(P)$.
\end{problem}
\begin{proof}
    对$P$在$G$中的任意一个共轭子群$Q$,存在$g\in G$使得\begin{align*}
        Q=gPg^{-1}\overset{\text{P<N}}{<}gNg^{-1}\xlongequal[]{N\lhd G}N,
    \end{align*}可见$P$在$G$的共轭子群都在$N$中,所以对$\forall g\in G,gPg^{-1}$是$N$的\Sl{p},并且

    (1)若$gPg^{-1}=P$,则\begin{align*}
        &g\in N_G(P)\subset N\cdot N_G(P)\\
        \Lra&G\subset N\cdot N_G(P).
    \end{align*}

    (2)若$gPg^{-1}\neq P$,注意到\begin{align*}
        &\text{$gPg^{-1}$是$P$在$G$中的共轭子群}\\
        &\text{$P$在$G$中的共轭子群也在$N$中},
    \end{align*}所以由$P$是$N$的\Sl{p}与推论\ref{p81tl1}可知\begin{align*}
        &\exists n\in N\st gPg^{-1}=nPn^{-1}\\
        \iff&(n^{-1}g)P(g^{-1}n)=P\\
        \iff&n^{-1}g\in N_G(P),
    \end{align*}记$t\in N_G(P)\st n^{-1}g=t$,就有\begin{align*}
        &g=nt\in N\cdot N_G(P)\\
        \xLra[]{g\text{的任意性}}&G\subset N\cdot N_G(P).
    \end{align*}

    由$N\cdot N_G(P)<G$与(1),(2)便知$G=N\cdot N_G(P)$.
\end{proof}
\begin{problem}[P98T35]
    已知$G$是有限群,$H<G,P$是$G$的\Sl{p},$p\Big||H|$,证明存在$a\in G\st aPa^{-1}\cap H$是$H$的\Sl{p}.
\end{problem}
\begin{proof}
    由$p\Big||H|$可知$H$存在\Sl{p}子群,记为$A$,它也是$G$的$p$-群,从而由Sylow第二定理(定理\ref{xldedl})可知\begin{align*}
        \exists a\in G\st A\subset aPa^{-1},
    \end{align*}从而\begin{align}
        A\subset aPa^{-1}\cap H.\label{fsdjkn}
    \end{align}

    注意到$aPa^{-1}\cap H$是$H$的阶为$p$的幂的子群,所以\begin{align}
        |aPa^{-1}\cap H|\leq|H|.\label{afsdkjn}
    \end{align}

    由\eqref{fsdjkn}与\eqref{afsdkjn}便知$aPa^{-1}$是$H$的\Sl{p},问题得证.
\end{proof}