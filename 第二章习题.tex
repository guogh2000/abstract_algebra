\section{CHAPTER2习题}
\begin{problem}[P97T1]
    已知$G$是有限群,$N\lhd G$,$\l(|N|,|G/N|\r)=1$.证明:若元素$a$的阶整除$|N|$,则$a\in N$.
\end{problem}
\begin{proof}
    考虑自然同态\begin{align*}
        \pi:G\to&G/N\\
        g\mapsto&gN,
    \end{align*}于是\begin{align*}
        \l[\pi(a)\r]^{o(a)}=&(aN)^{o(a)}\\
        \xlongequal[]{N\lhd G}&a^{o(a)}N^{o(a)}\\
        =&N,
    \end{align*}又$N$是$G/N$的幺元,所以\begin{align}\label{snjdkk}
        \l|\pi(a)\r|\Big|\l|o(a)\r|\Big|\l|N\r|.
    \end{align}

    注意到\begin{align*}
        aN\in G/N,
    \end{align*}所以\begin{align*}
        \l|aN\r|\Big|\l|G/N\r|,
    \end{align*}即\begin{align}\label{asjgndk}
        \l|\pi(a)\r|\Big|\l|G/N\r|.
    \end{align}

    由式\eqref{snjdkk}与\eqref{asjgndk}结合$\l(|N|,|G/N|\r)=1$可知\begin{align*}
        &|\pi(a)|=1\\
        \Lra&\pi(a)=N\\
        \Lra&aN=N\\
        \Lra&a\in N,
    \end{align*}问题得证.
\end{proof}
\begin{problem}[P97T2]
    设$c$是群$G$中阶为$rs$的元素,其中$(r,s)=1$.证明$c$可以表示成$c=ab$,其中$a$的阶为$r,b$的阶为$s$,且$a,b$都是$c$的方幂.
\end{problem}
\begin{proof}
    由$(r,s)=1$可知\begin{align*}
        \exists u,v\in\Z\st ur+vs=1,
    \end{align*}不难验证\begin{align*}
        &a=c^{vs}\\
        &b=c^{ur}
    \end{align*}满足题设要求.
\end{proof}
\begin{problem}[P97T3]
    已知群$G$中元素$a$的阶与正整数$k$互素,证明方程$x^k=a$在$<a>$内恰有一解.
\end{problem}
\begin{proof}
    由$\l(o(a),k\r)=1$可知\begin{align*}
        \exists u,v\in \Z\st uo(a)+vk=1,
    \end{align*}所以\begin{align*}
        a^{uo(a)+vk}=a,
    \end{align*}即\begin{align*}
        a^{vk}=a,
    \end{align*}由此可见$x=a^v$是$x^k=a$在$<a>$内的解.

    现设$x=a^v_0$也是$x^k=a$在$<a>$内的解,其中$0\leq v-v_0\leq o(a)-1$,于是\begin{align*}
        &a^{vk}=a\\
        &a^{v_0k}=a,
    \end{align*}即\begin{align*}
        vk\equiv1&\mod o(a)\\
        v_0k\equiv1&\mod o(a),
    \end{align*}所以\begin{align*}
    o(a)|(v-v_0)k,
    \end{align*}由此可见$v-v_0=0$即$v=v_0$.

    综上,问题证毕.
\end{proof}
\begin{problem}[P97T4]
    证明在群中,$ab$与$ba$有相同的阶.
\end{problem}
\begin{proof}
    注意到$ab=b^{-1}\cdot ba\cdot b$,所以\begin{align*}
        &ab=e\\
        \iff&b^{-1}\cdot ba\cdot b=e\\
        \iff&ba=e,
    \end{align*}可见问题得证.
\end{proof}
\begin{problem}[P97T10]
    证明$S_n$中的任意一个置换能由$n-1$个对换$(12),(13),\cdots,(1n)$生成,也能由$n-1$个对换$(12),(23),\cdots,(n-1\ n)$生成.
\end{problem}
\begin{proof}
    由引理\ref{p65yl}可知$S_n$中的任意一个置换都可以拆成若干个对换$(ij)$的复合,注意到\begin{align*}
        (1i)(1j)(1i)=(ij),
    \end{align*}所以任意一个置换都能由$(n-1)$个对换\begin{align*}
        (12),(13),\cdots,(1n)
    \end{align*}生成,第一个论断证毕.

    注意到\begin{align*}
        (1i)(i\ i+1)(1i)=(1\ i+1),
    \end{align*}故可用归纳法证明\begin{align*}
        (12),(23),\cdots,(n-1\ n)
    \end{align*}能生成\begin{align*}
        (12)(13),\cdots,(1n),
    \end{align*}再结合第一个论断可知它能生成$S_n$中的任意一个置换,第二个论断证毕.
\end{proof}
\begin{problem}[P97T16]
    设$H_1,H_2$是群$G$的两个子群,证明$H_1\cap H_2$的任一左陪集是$H_1$的一个左陪集与$H_2$的一个左陪集的交.
\end{problem}
\begin{proof}
    若能证明对$\forall g\in G$都有$g(H_1\cap H_2)=gH_1\cap gH_2$,则问题得证.证明思路为两者互相包含.

    (1)对$\forall$给定的$g\in G$,在$g(H_1\cap H_2)$中任取$gh_0$,有\begin{align*}
        &h_0\in H_1\cap H_2\\
        \Lra&gh_0\in gH_1,gh_0\in gH_2\\
        \Lra&gh_0\in gH_1\cap gH_2\\
        \Lra&g(H_1\cap H_2)\subset gH_1\cap gH_2.
    \end{align*}

    (2)反之,对任意给定的$g\in G$,在$gH_1\cap gH_2$中任取$gh_0$,有\begin{align*}
        &gh_0\in gH_1,gh_0\in gH_2\\
        \overset{g\text{的任意性}}{\Lra}&h_0\in H_1,h_0\in H_2\\
        \Lra&h_0\in H_1\cap H_2\\
        \Lra&gh_0\in g(H_1\cap H_2)\\
        \Lra&gH_1\cap gH_2\subset g(H_1\cap H_2).
    \end{align*}

    由(1),(2)便知$g(H_1\cap H_2)=gH_1\cap gH_2$,综上,问题得证.
\end{proof}
\begin{problem}[P98T18]\label{p98t18}
    设$G$为有限群,$H<G$且$|G/H|=n>1$.证明$G$或者含有指数能整除$n!$的非平凡正规子群,或者$G$同构于$S_n$的一个子群.
\end{problem}
\begin{proof}
    记$P=\l\{a_iH:i=,1,\cdots,n,a\in G\r\}$为$H$的所有左陪集组成的集合.定义群$G$在集合$P$上的群作用为\begin{align*}
        g:P\to&P\\
        a_iH\mapsto&ga_iH,
    \end{align*}容易验证这是良定义的.从而群作用$g$能诱导出群同态\begin{align*}
        \phi: G\to S_n,
    \end{align*}其中\begin{align*}
        \ker\phi\lhd G
    \end{align*}且\begin{align}
        &\im\phi<S_n\nonumber\\
        \overset{|G/H|>1}{\Lra}&|\im\phi|>1\label{skdnj}\\
        \Lra&\im\phi\neq\l\{e\r\}\nonumber\\
        \Lra&\ker\phi\neq G\label{jnakd}.
    \end{align}

    (1)当$\ker\phi\neq\l\{e\r\}$时,结合\eqref{jnakd}可知$G/\ker\phi$中的任意元素$g\ker\phi$满足\begin{align*}
        &g\ker\phi\lhd G\\
        &g\ker\phi\neq\l\{e\r\}\\
        &g\ker\phi\neq G,
    \end{align*}又\begin{align*}
        |g\ker\phi|\Big||G/\ker\phi|\xlongequal[]{\text{群同态基本定理(定理\ref{qttjbdy})}}|\im\phi|\Big||S_n|=n!,
    \end{align*}所以$g\ker\phi$是$G$的一个指数能整除$n!$的非平凡子群.
    
    (2)若$\ker\phi=\l\{e\r\}$,则\begin{align*}
        &G/\ker\phi\cong\im\phi\\
        &G\cong\im\phi<S_n,
    \end{align*}结合\eqref{skdnj}可知此时$G$同构于$S_n$的非平凡的正规子群$\im\phi$.

    由(1),(2)便知问题证毕.
\end{proof}
\begin{problem}[P98T19]
    设$G$为有限群,$p$是$|G|$的最小素因子.证明指数为$p$的子群(若存在)必正规.
\end{problem}
\begin{proof}
    设$H$是群$G$的子群并且$|H|=p$,下面来证明$H\lhd G$.记$G$在$H$上的传递置换表示为$\phi$.

    (1)若$\ker\phi=\l\{e\r\}$,则由群同态基本定理(定理\ref{qttjbdy})可知\begin{align*}
        &G/\ker\phi\cong\im\phi\\
        \iff&G\cong\im\phi<S_p,
    \end{align*}于是\begin{align*}
        |G|=|\im\phi|\Big||S_p|=p!,
    \end{align*}结合$p$是$|G|$的最小素因子便知$|G|=p$,于是\begin{align*}
        &|H|=\frac{|G|}{|G/H|}=\frac{p}{p}=1\\
        \iff&H=\l\{e\r\}\\
        \Lra\ &H\text{是$G$的正规子群}.
    \end{align*}

    (2)若$\ker\phi\neq\l\{e\r\}$,





\end{proof}
\begin{problem}[P98T21]
    证明任一非交换的$6$阶群同构于$S_3$.
\end{problem}
\begin{proof}
    \stars
\end{proof}
\begin{problem}[P98T29]
    设$G$为有限群,$A,B$是$G$的两个非空子集,如果$|A|+|B|>|G|$,则$AB=G$.
\end{problem}
\begin{proof}
    \stars
\end{proof}
\begin{problem}[P98T33]
    设$G$是有限群,$N\lhd G,P$为$N$的一个$\mr{Sylow}$子群,$T$为$P$在$G$中的正规化子.证明$G=NT$.
\end{problem}
\begin{proof}
    \stars
\end{proof}
\begin{problem}[P98T35]
    设$G$为有限群,$H<G,P$是$G$的$\Sp$子群,$N\lhd G$.证明$P\cap N$是$N$的一个$\Sp$子群,同时$PN/N$是$G/N$的一个$\Sp$子群.
\end{problem}
\begin{proof}
    \stars
\end{proof}