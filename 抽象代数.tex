\documentclass[b5paper]{book}









\usepackage{ctex}
\usepackage{amsmath}
\usepackage{extarrows}
\usepackage{amssymb}
\usepackage{amsthm}
\usepackage{tikz-cd}
\usepackage{hyperref}
% \usepackage[hidelinks]{hyperref}
% \usepackage[notref,notcite]{showkeys}
\newcommand{\xl}{\overrightarrow}
\newcommand{\ys}{\mathcal}
\newcommand{\gs}{\texorpdfstring}
\newcommand{\arccot}{\mathrm{arccot}\,}
\newcommand{\sech}{\mathrm{sech}\,}
\newcommand{\p}{\partial}
\newcommand{\e}{\mathrm{e}}
\newcommand{\N}{\mathbb{N}}
\newcommand{\Z}{\mathbb{Z}}
\newcommand{\I}{\mathbb{I}}
\newcommand{\R}{\mathbb{R}}
\newcommand{\Q}{\mathbb{Q}}
\newcommand{\Var}{\mathrm{Var}}
\newcommand{\Cov}{\mathrm{Cov}}
\newcommand{\Lla}{\Longleftarrow}
\newcommand{\Lra}{\Longrightarrow}
\newcommand{\lla}{\longleftarrow}
\newcommand{\lra}{\longrightarrow}
\newcommand{\llas}{\leftleftarrows}
\newcommand{\rras}{\rightrightarrows}
\newcommand{\mr}{\mathrm}
\newcommand*{\im}{\mathrm{im}\,}
\newcommand{\st}{\ \ \mathrm{s.t.}\ \ }
\newcommand{\stars}{\text{*****************************************************}}
\newcommand{\os}{\texorpdfstring}
\newcommand{\Aut}{\mathrm{Aut}\,}
\newcommand{\In}{\mathrm{In}\,}
%右推出、左推不出
\newcommand\rtc{\mathrel{\vcenter{\ialign{##\crcr$\Rightarrow$\crcr
\noalign{\nointerlineskip\vskip2pt}$\nLeftarrow$\crcr}}}}
%左推出、右推不出
\newcommand\ltc{\mathrel{\vcenter{\ialign{##\crcr$\nRightarrow$\crcr\noalign{\nointerlineskip\vskip2pt}$\Leftarrow$\crcr}}}}
\newcommand{\Sp}{\mr{Sylow}\ p\text{-}}
\renewcommand{\theequation}{\thesection.\arabic{equation}}
\renewcommand{\d}{\mathrm{d}}
\renewcommand{\l}{\left}
\renewcommand{\r}{\right}
\renewcommand{\epsilon}{\varepsilon}
\renewcommand{\phi}{\varphi}
\renewcommand{\leq}{\leqslant}
\renewcommand{\geq}{\geqslant}
\renewcommand{\emptyset}{\varnothing}
\DeclareMathAlphabet{\mathcal}{OMS}{cmsy}{m}{n}
\let\mathbb\relax
\DeclareMathAlphabet{\mathbb}{U}{msb}{m}{n}
\everymath{\displaystyle}










\newtheorem{theorem}{\indent 定理}[section]
\newtheorem{lemma}{\indent 引理}[section]
\newtheorem{proposition}{\indent 命题}[section]
\newtheorem{corollary}{\indent 推论}[section]
\newtheorem{definition}{\indent 定义}[section]
\newtheorem{example}{\indent 例}[section]
% \newtheorem{example_addition}{\indent 补充题}
\newtheorem{remark}{\indent 注}[section]
\newtheorem{problem}{\indent 问题}[chapter]
\newtheorem*{analysis}{\indent 分析}
% \newtheorem{*problem}[problem]{\indent 问题*}
% \newtheorem{exercise}{\indent 练习题}
\newenvironment{solution}{\begin{proof}[\indent 解]}{\end{proof}}
\renewcommand{\proofname}{\indent 证}


% \title{抽象代数}
% \author{郭冠华}
% \date{\today}



\begin{document}
\allowdisplaybreaks
\tableofcontents
\raggedbottom
    \chapter{代数基本概念}
\section{代数运算}
\begin{definition}[代数运算]
    设$A$是一个非空集合,任意一个由$A\times A\lra A$的映射就称为定义在$A$上的代数运算.
\end{definition}
\section{群的定义和简单性质}
\begin{definition}[群]
    设$G$是一个非空集合,在$G$上定义了一个称之为乘法的代数运算,记作$ab$,若该代数运算满足如下性质,就称$G$为一个群\begin{align*}
        [\text{结合律}]&(1)(ab)c=a(bc);\\
        [\text{左幺元}]&(2)\exists e\in G\ \ \mr{s.t.}\ \ \forall a\in G,\text{有}ea=a;\\
        [\text{左逆元}]&(3)\forall a\in G,\exists b\in G\ \ \mr{s.t.}\ \ ba=e.
    \end{align*}
\end{definition}
1.若$ba=e$,则$ab=e$.\begin{proof}
    任取$b\in G$,存在$c\in G$,使得$cb=e$,于是\begin{align}
        a=ea=(cb)a=c(ba)=ce,\label{1.1}
    \end{align}在等式\eqref{1.1}两侧同时右乘$b$,就有\begin{align*}
        ab=(ce)b=c(eb)=cb=e,
    \end{align*}问题证毕.
\end{proof}

2.若对所有的$a\in G$,有$ea=a$,那么也有$ae=a$,对所有的$a\in G$.\begin{proof}
    取$b\in G$,使得$ba=e$,同时$ab=e$,于是\begin{align*}
        ae=a(ba)=(ab)a=ea=a,
    \end{align*}问题证毕.
\end{proof}
3.群$G$中有惟一的元素$e$具有性质\begin{align*}
    \forall a\in G,ea=ae=a.
\end{align*}\begin{proof}
    假设$G$中有元素$e_1,e_2$满足此性质,则\begin{align*}
        e_1=e_1e_2=e_2,
    \end{align*}可见惟一性得证.
\end{proof}
4.对群$G$中任意元素$a$,有惟一元素$b$,使$ab=ba=e$.\begin{proof}
    假设在$G$中还有元素$c$满足$ac=ca=e$,则\begin{align*}
        c=ec=ce=c(ab)=(ca)b=eb=b,
    \end{align*}这就证明了惟一性.
\end{proof}
5.对于群$G$中任意元素$a,b$,方程\begin{align*}
    ax=b
\end{align*}在$G$中有惟一解.\begin{proof}
    在题设方程两侧同时左乘$a^{-1}\in G$,有\begin{align*}
        (a^{-1}a)x=a^{-1}b
    \end{align*}亦即$x=a^{-1}b\in G$,解的存在性得证.

    假设还有元素$c\in G$满足$ac=b$,则\begin{align}
        ax=b=ac,\label{5}
    \end{align}在等式\eqref{5}两侧同时左乘$a^{-1}$,就有\begin{align*}
        (a^{-1}a)x=(a^{-1}a)c\iff x=c,
    \end{align*}解的惟一性得证.

    综上所述,问题得证.
\end{proof}
\begin{definition}[Abel群(或交换群)]
    若群$G$的运算适合交换律,则称群$G$为Abel群(或交换群).
\end{definition}
\begin{definition}[阶]
    群$G$中所含元素的个数称为群$G$的阶,记作$|G|$.
\end{definition}
\begin{definition}[有限群与无限群]
    若$|G|$是一个有限数(无限数),则称群$G$为有限群(无限群).
\end{definition}
\section{群的例子}
本节将不加证明地给出一些常见的群的例子和性质.
\begin{definition}[图形\os{$F$}的对称群,二面体群]\label{txfddiq}\label{emtq}
    已知$F$是平面上的一个图形.令$G_F$为全体保持$F$不变的平面正交变换所成的集合,则$G_F$在变换的称发下成群,称为图形$F$的对称群.

    若用$T$表示绕$O$旋转$90^\circ,S$表示对于直线$l$的镜面反射,则不难看出\begin{align*}
        G_F=\l\{T,T^2,T^3,T^4,ST,ST^2,ST^3,ST^4\r\},
    \end{align*}其中$T^4=I,S^2=I,ST=T^{-1}S$($I$表示恒等映射).
    
    类似地,若$F$是平面上正$n$边形,则$F$的对称群$G_F$由$2n$个元素组成.令$T$为绕中心转$\frac{2\pi}{n},S$为对于某一对称轴的镜面反射,则有\begin{align*}
        G_F=\l\{T,T^2,\cdots,T^n,ST,ST^2,\cdots,ST^n\r\},
    \end{align*}其中$T^n=I,S^2=I,ST=T^{-1}S$.称这些群为二面体群记作$D_n$.
\end{definition}
\begin{definition}[对称群\os{$S_n$},集合\os{$M$}的全变换群,\os{$n$}元置换,\os{$n$}元对称群,不相交的]\label{diqsn}\label{jhmdqbhq}\label{vh}\label{nydiq}\label{diq}\label{lh}\label{dh}
    若$M$为非空集合,则$M$到自身的全体可逆变换关于变换的乘法成群,称该群为集合$M$的全变换群,记作$S(M)$.当$M$是无限集,$S(M)$为无限群.

    当$M$含有$n$个元素时,$M$的可逆变换称为$M$的$n$元置换,$S(M)$称为$n$元对称群,简记为$S_n$.

    若$M$中的元素用$1,2,\cdots,n$编号后,则$S$中的元素表示为\begin{align*}
        \sigma=\l(\begin{matrix}
            1&2&&n\\
            \alpha_1&\alpha_2&\cdots&\alpha_n,
        \end{matrix}\r)
    \end{align*}其中$\alpha_i=\sigma(i),i=1,2,\cdots,n$.易见$n$元置换与$n$阶排列之间存在一一对应,亦即$|S|=n!$.

    若一个$n$元置换$\sigma$将$1,2,\cdots,n$中某$m$个数$\alpha_1,\cdots,\alpha_m$轮换,即\begin{align*}
        &\sigma(\alpha_1)=\alpha_2,\sigma(\alpha_2)=\alpha_3,\cdots\\
        &\sigma(\alpha_{m-1})=\alpha_m,\sigma(\alpha_m)=\alpha_1,
    \end{align*}其余的数保持不变,则称$\sigma$为轮换,表示为\begin{align*}
        \sigma=(\alpha_1\alpha_2\cdots\alpha_m).
    \end{align*}当$m=2$时,也称$\sigma$为对换.

    若$S_n$中的两个轮换$(\alpha_1\alpha_2\cdots\alpha_m)$与$\beta_1\beta_2\beta_l$满足\begin{align*}
        \alpha_i\neq\beta_j,i=1,2,\cdots,m,j=1,2,\cdots,l,
    \end{align*}则称这两个对换为不相交的.
\end{definition}
\begin{proposition}
    非单位的置换能惟一地表成一些不相交的轮换的乘积.\qed
\end{proposition}
\section{子群,陪集}
\begin{definition}[子群]
    若群$G$的非空子集合$H$对$G$的运算也成群,则称群$H$是群$G$的子群,记作$H<G$.
\end{definition}
\begin{theorem}\label{p31dl1}
    群$G$的非空子集合$H$是群$G$的子群的充分必要条件是\begin{align*}
        \forall a,b\in H\Lra ab^{-1}\in H.
    \end{align*}
\end{theorem}
\begin{proof}
    必要性显然,接下来证明充分性.

    (1)结合律:显然满足;

    (2)幺元的存在性:$\forall a\in H$,取$b=a$,则$e=aa^{-1}\in H$;

    (3)逆元的存在性:$\forall b\in H$,取$a=e$,则$b^{-1}=eb^{-1}\in H$.

    结合(1),(2),(3)可得$H$成为群,进而是群$G$的子群.
\end{proof}
\begin{definition}[左陪集,右陪集]
    设群$H$是群$G$的一个子群,对$G$中的任意一个元素$a$,称$aH=\l\{ah:h\in H\r\}$是$H$的一个左陪集;称$Ha=\l\{ha:h\in H\r\}$是$H$的一个右陪集.
\end{definition}
\begin{definition}[基数]\label{ju}
    若两集合之前存在一个一一对应,则称这两个集合有相同的基数.对任意集合$X$,记$X$的基数为$|X|$.

    当$X$为无限集时,记$|X|=\infty$;当$X$为有限集时,记$|X|$为$X$所含元素的个数.
\end{definition}
\begin{definition}[商集,指数]\label{uj}\label{vs}
    称群$G$关于子群$H$的所有左陪集(或右陪集)组成的集合为群$G$关于子群$H$的左商集(或右商集),称它的基数为$H$在$G$中的指数,记作$[G:H]$.
\end{definition}
\begin{theorem}\label{vu1}
    设$G$是群,$H<G$,则$H$的任意一个左陪集$gH$与$H$含有同样多的元素.该定理对于右陪集同样成立.
\end{theorem}
\begin{proof}
    易见$h\mapsto ah$是子群$H$到左陪集$aH$的一个一一对应,$h\mapsto ha$是子群$H$到右陪集$Ha$的一个一一对应,因此定理得证.
\end{proof}
\begin{theorem}\label{p31dy2}
    设群$H$是群$G$的子群.$H$的任意两个左(右)陪集要么相等,要么无公共元素.群$G$可以表示为若干个不相交的左(右)陪集之并.
\end{theorem}
\begin{proof}
    利用相互包含证明第一个论断:取$H$的两个左陪集$aH,bH$并假设它们有公共元素,即有$ah_1\in aH,bh_2\in bH$满足\begin{align}
        ah_1=bh_2,\label{7}
    \end{align}等式\eqref{7}两端同时右乘$h_1^{-1}$,有\begin{align*}
        a=bh_2h_1^{-1}\in bH,
    \end{align*}可见$aH\subset bH$.同理可证$aH\supset bH$,进而$aH=bH$.第一个论断证毕.

    第二个论断的证明:由于$a\in aH$,所以\begin{align*}
        G=\bigcup_{a\in G}aH,
    \end{align*}去掉其中的重复项,就有\begin{align*}
        G=\bigcup_{\alpha}a_{\alpha}H,
    \end{align*}其中$a_{\alpha}H$两两无交.
\end{proof}
\begin{corollary}[Lagrange定理]\label{Lagrangedl}
    设$G$是有限群,$H$是它的子群,则$|H|$是$|G|$的因子.
\end{corollary}
\begin{proof}
    设$|G|=n,|H|=t$,由定理\ref{p31dy2}可得\begin{align}
        G=a_1H\cup a_2H\cup\cdots\cup a_rh,\label{xx}
    \end{align}其中$a_iH\cap a_jH=\varnothing(i,j=1,2,\cdots,r\text{且}i\neq j)$,在等式\eqref{xx}两侧同时取因子,并利用定理\ref{vu1}就有\begin{align*}
        |G|=r|H|,
    \end{align*}从而$|H|$是$|G|$的因子.
\end{proof}
\begin{definition}[由$a$生成的子群]
    在群$G$中,任意一个元素$a$的全体方幂组成的集合$\l\{a^m:m\in\Z\r\}$显然成$G$的子群,称为由$a$生成的子群.
\end{definition}
\begin{remark}
    (1)元素$a$的方幂要么两两不同要么存在$l\in\Z_+$使得$a^l=e$;

    (2)在(1)的后一种情形中,一定有最小的正整数$d$满足$a^d=e$.此时将$d$称为元素$a$的阶.
\end{remark}
\begin{corollary}
    设$G$为一有限群,则$G$中每一个元素的阶一定是$|G|$的因子.
\end{corollary}
\begin{proof}
    设$H$是由$G$中的元素$a$生成的子群,则\begin{align*}
        (\text{i})&|a|=|<a>|=|H|;\\
        (\text{ii})&H\text{是}G\text{的子群}\Lra|H|\text{整除}|G|,
    \end{align*}可见$G$中每一个元素的阶一定是$G$的因子.
\end{proof}
\section{群的同构}
\begin{definition}[群的同构]
    若$G,G'$是两个群,$\varphi:g\mapsto g',G\lra G'$是一一对应,并且满足$\forall g_1,g_2\in G$\begin{align}
        \varphi(g_1g_2)=\varphi(g_1')\varphi(g_2'),\label{1.5.1}
    \end{align}则称群$G$同构于群$G'$,记作$G\cong G'$.适合等式\eqref{1.5.1}的一一对应称为同构映射,简称同构.
\end{definition}
\begin{lemma}\label{yl1.5.1}
    任意非空集合上的全体可逆变换构成的集合关于变换的乘法成群.\qed
\end{lemma}
\begin{theorem}[Cayley定理]
    任何一个群都同构于某一集合上的变换群.
\end{theorem}
\begin{proof}
    设$G$是群.对每一个$a\in G$,定义$G$上的变换$\varphi_a$如下\begin{align*}
        \varphi_a(x)=ax,x\in G,
    \end{align*}可见$\forall x\in G$\begin{align*}
        (\text{i})&\varphi_{a^{-1}}\varphi_a(x)=\varphi_{a^{-1}}(ax)=a^{-1}ax=x;\\
        (\text{ii})&\varphi_a\varphi_{a^{-1}}(x)=\varphi_a(a^{-1}x)=aa^{-1}x=x,
    \end{align*}可见$\forall a\in G,\varphi_a$均是可逆变换.记$G_l=\l\{\varphi_a:a\in G\r\}$,于是$\forall a,b\in G_l$\begin{align*}
        \varphi_a\varphi_{b^{-1}}(x)=\varphi_a(b^{-1}x)=ab^{-1}x=\varphi_{ab^{-1}}(x),
    \end{align*}即$\varphi_a\varphi_{b^{-1}}=\varphi_{ab^{-1}}\in G_l$,根据引理\ref{yl1.5.1}与定理\ref{p31dl1}可得$G_l$成群,亦即$G_l$是一变换群.

    根据$G_l$定义易知映射$a\mapsto\varphi_a$为满映射.
    
    由于\begin{align*}
        \varphi_a(e)=a,
    \end{align*}所以当$a\neq b$时,$\varphi_a\neq\varphi_b$,亦即映射$a\mapsto\varphi_a$是单映射.进而映射$a\mapsto\varphi_a$是一一对应.再由$\varphi_a\varphi_b=\varphi_{ab}$可知所述映射为同构映射,从而$G\cong G_l$,定理得证.
\end{proof}
\section{同构,正规子群}
\begin{definition}[同态映射]
    若$\varphi$是群$G$到群$G'$的映射,满足$\forall g_1,g_2\in G$\begin{align*}
        \varphi(g_1g_2)=\varphi(g_1)\varphi(g_2)
    \end{align*}则称$\varphi$是群$G$到$G'$的同态映射,或同态.
\end{definition}
\begin{remark}
    在同态映射的定义中,既不要求它是映上的,也不要求它是单射.
\end{remark}

当$\varphi$是$G$到$G'$的同态映射时,常常简记为\begin{align*}
    \varphi:G\mapsto G'.
\end{align*}

\begin{definition}[象]
    若$\varphi:G\mapsto G'$,定义\begin{align*}
        \varphi G=\l\{\varphi(a):a\in G\r\}
    \end{align*}为同态映射$\varphi$的象.
\end{definition}
\begin{remark}
    (1)易见$\varphi G$是$G'$的子群;

    (2)若$\varphi$是映上的,即$\varphi G=G'$,称$\varphi$为满同态;

    (3)若$\varphi$是单射,即$G$与$\varphi G$同构,亦即$G$与$G'$的一个子群同构,则称$\varphi$为单一同态,或嵌入映射.
\end{remark}
\begin{definition}[完全反象,核]
    对于同态映射$\varphi:G\mapsto G'$,定义\begin{align*}
        \varphi^{-1}(a')=\l\{a:\varphi(a)=a'\r\}
    \end{align*}为元素$a'$的完全反象.特别地,定义$\varphi^{-1}(e')$为同态映射$\varphi$的核,记作$\ker{\varphi}$.
\end{definition}
\begin{proposition}\label{xz1.6.1}
    记$\varphi(a)=a'$,则$\varphi^{-1}(a')=\l\{\begin{matrix}
        a\ker{(\varphi)};\\
        \ker{(\varphi)}a.
    \end{matrix}\r.$
\end{proposition}
\begin{proof}
    (1)任取$h\in\ker\varphi$,有\begin{align*}
        \varphi(ah)\xlongequal[]{\text{同态映射}}\varphi(a)\varphi(h)=a'e'=a',
    \end{align*}这说明$a\ker\varphi$中的元素在映射$\varphi$下的象均为$a'$,亦即$a\ker\varphi\subset\varphi^{-1}(a')$;

    (2)反之,任取$a\in\varphi^{-1}(a')$,即$\varphi(a)=a'$.又$e\in\ker\varphi$,从而\begin{align*}
        a=ae\in a\ker\varphi,
    \end{align*}这说明在映射$\varphi$下的象为$a'$的元素在$a\ker\varphi$中,亦即$a\ker\supset\varphi^{-1}(a')$.

    由(1),(2)可知$\varphi^{-1}(a)=a\ker(\varphi)$.

    同理可证$\varphi^{-1}(a')=\ker(\varphi)a$.
\end{proof}
\begin{definition}[正规子群]\label{sadjfgnka}
    设群$H$是群$G$的子群,若对任意$g\in G$,都有$gH=Hg$,则称$H$是$G$的正规子群,记作$H\lhd G$.
\end{definition}
\begin{remark}
    (1)由命题\ref{xz1.6.1}可知,同态的核都是正规子群;

    (2)正规子群的定义可以改写为\begin{align*}
        \forall g\in G,gHg^{-1}=H.
    \end{align*}正规子群的定义换个说法就是子群$H$的左右陪集相等;

    (3)在Abel群中,每个子群都正规.
\end{remark}
\section{商群}
\begin{definition}[群的子集合的运算]
    \ 

    1.定义\begin{align*}
        AB=\l\{ab|a\in A,b\in B\r\},
    \end{align*}子集乘积满足结合律:$(AB)C=A(BC)$;

    2.定义\begin{align*}
        A^{-1}=\l\{a^{-1}|a\in A\r\}.
    \end{align*}
\end{definition}

利用集合运算,定理\ref{p31dl1}可改写为\begin{align*}
    \text{群$G$的非空子集合$H$是子群}\iff HH^{-1}\subset H.
\end{align*}
\begin{theorem}
    设$H$是群$G$的一个子群.$H$是正规子群$\iff H$的任意两个左(右陪集)之积还是左(右陪集).
\end{theorem}
\begin{proof}
    (1)必要性

    任取正规子群$H$的两个左陪集$aH$与$bH$,有\begin{align*}
        (aH)(bH)=a(Hb)H=a(bH)H=(ab)(HH)=abH,
    \end{align*}必要性得证;

    (2)充分性

    任取$H$的两个左陪集$aH$与$bH$,根据已知条件可设$(aH)(bH)=cH$,由于$ab\in(aH)(bH)$,所以$ab\in cH$,再由$ab\in abH$与定理\ref{p31dy2}可得\begin{align}
        abH=cH=(aH)(bH),
    \end{align}等式两端同时左乘$a^{-1}$,有\begin{align*}
        bH=HbH\supset Hbe=Hb,
    \end{align*}由于$b$具有任意性,故可以将其改成$b^{-1}$,得到\begin{align*}
        b^{-1}H\supset Hb^{-1},
    \end{align*}等式两边同时左乘$b$,右乘$b$,得到\begin{align*}
        Hb\supset bH,
    \end{align*}亦即$bH=Hb$,可见$H$是正规子群.
\end{proof}
令$G/H$代表正规子群$H$的全部不同的右陪集组成的集合.
\begin{proposition}
    $G/H$在陪集的运算下成群.
\end{proposition}
\begin{proof}
    (1)结合律

    由$(Ha)(Hb)=Hab$可见,陪集之间的乘法可归结为陪集代表的乘法,故结合律显然成立;

    (2)左幺元

    $\forall Ha\in G/H,$有\begin{align*}
        H\cdot Ha=Ha,
    \end{align*}可见左幺元存在,为$H$;

    (3)左逆元

    $\forall Ha\in G/H$,有\begin{align*}
    (Ha^{-1})(Ha)=H(a^{-1}H)a=H(Ha^{-1})a=(HH)(a^{-1}a)=H,
    \end{align*}可见$G/H$中的任一元都有左逆元.

    (1),(2),(3)说明$G/H$成群,问题得证.
\end{proof}
\begin{definition}[商群]
    $G/H$在陪集的乘法下所成的群称为群$G$对正规子群$H$的商群,仍记作$G/H$.
\end{definition}
\begin{proposition}\label{asjdga}
    设群$H$是群$G$的正规子群,定义映射\begin{align*}
        \varphi:G\to&G/H\\
        g\mapsto&Hg,
    \end{align*}则$\varphi$是满同态且$\ker\varphi=H$.
\end{proposition}
\begin{proof}
    (1)$\forall a,b\in G$,有\begin{align*}
        &\varphi(ab)\\
        =&Hab\\
        =&HabH\\
        =&Ha(bH)\\
        =&Ha(Hb)\\
        =&HaHb\\
        =&\varphi(a)\varphi(b)\\
        \Lra&\varphi\text{是同态映射};
    \end{align*}

    (2)根据商群的定义,$\varphi$显然是映上的;

    (3)对$\forall h\in H$,注意到\begin{align*}
        &\varphi(h)\\
        =&hH\\
        =&H\\
        =&eH,
    \end{align*}可见$h\in H$,于是$\ker\varphi\supset H$.同时对$\forall k\in\ker\varphi$,有\begin{align*}
        \varphi(k)=&kH\\
        =&H,
    \end{align*}所以对任意$h\in H$,都有$kh\in H$,现取$h=e$,所以\begin{align*}
        k=ke\in H,
    \end{align*}即$k\in H$,所以$\ker\varphi\subset H$.

    (1),(2)说明$\varphi:G\to G/H$为满同态;(3)说明$\ker\varphi=H$.
\end{proof}
\begin{remark}
    由于$H\lhd  G$,所以若定义\begin{align*}
        \varphi:G\to&G/H\\
        g\mapsto&gH,
    \end{align*}则命题\ref{asjdga}也成立.
\end{remark}
\begin{definition}[自然同态]
    称命题\ref{asjdga}中的$\varphi$为$G\to G/H$的自然同态.
\end{definition}
\begin{remark}
    由命题\ref{xz1.6.1}可知,同态的核都是正规子群;自然同态的构造说明每个正规子群也都是某一同态的核.
\end{remark}
\begin{lemma}\label{jskdf}
    若$H$为群$G$的子群,$a,b\in G$,则\begin{align*}
        b^{-1}a\in H\iff aH=bH;\\
        ab^{-1}\in H\iff Ha=Hb.
    \end{align*}
\end{lemma}
\begin{proof}
    只要证明第一条即可,第二条同理可证.
    
    (1)必要性

    可设$h\in H$满足$b^{-1}a=h$,从而$a=bh\in bH$,又$e\in H$且$a=ae$,故\begin{align*}
        a=ae\in aH\\
        a\in bH,
    \end{align*}可见$aH\cap bH\neq\varnothing$,进而$aH=bH$,必要性得证;

    (2)充分性

    等式$aH=bH$两端同时左乘$b^{-1}$有\begin{align*}
        b^{-1}aH=H\Lra b^{-1}a\cdot e\in H\iff b^{-1}a\in H,
    \end{align*}充分性得证.
\end{proof}
\begin{theorem}[群同态基本定理]\label{qttjbdy}
    若$\sigma:G\to G'$,则$G/\ker\sigma\cong\im{\sigma}$.进一步,若$\sigma$是满同态,则$G/\ker\sigma\cong G'$.
\end{theorem}
\begin{proof}
设$\varphi:G\to G/\ker{\sigma}$是自然同态,则得到两个满同态$\sigma$和$\varphi$,交换图如下:\begin{center}
        \begin{tikzcd}
            G \arrow[rr, "\sigma"] \arrow[d, "\varphi"'] &  & \im{\sigma} \\
            G/\ker{\sigma} \arrow[rru, "\psi"', dashed]             &  &   
            \end{tikzcd}
    \end{center}其中虚线部分的$\psi$表示我们要找的同构.定义映射\begin{align*}
        \psi_0(\ker\sigma\cdot a)=\sigma(a),
    \end{align*}显然$\psi_0$是良定义的.

    由于\begin{align*}
        \psi_0(\ker\sigma\cdot a\ker\sigma \cdot b)\xlongequal[]{\ker\sigma\text{是正规子群}}\psi_0(\ker\sigma\cdot ab)=\sigma(ab)=\sigma(a)\sigma(b),
    \end{align*}所以$\psi_0$是同态映射.
    
    当$\sigma(a)=\sigma(b)$时,有\begin{align*}
        \sigma(a)(\sigma(b))^{-1}=&e'\\
        \sigma{(ab^{-1})}=&e',
    \end{align*}根据引理\ref{jskdf},\begin{align*}
        ab^{-1}\in\ker\sigma\iff b^{-1}\ker\sigma=a^{-1}\ker\sigma,
    \end{align*}即$a\ker\sigma=b\ker\sigma$,亦即$\ker\sigma\cdot a=\ker\sigma\cdot b$.可见$\psi_0$为单射.

    显然$\psi_0$是满射.

    综上所述,$\psi_0$是同构映射.取$\psi=\psi_0$即证明了$G/\ker\sigma\cong\in\sigma$.

    进一步,若$\sigma$是满同态,则$G'\cong\im\sigma$,从而$G/\ker\sigma\cong G'$.
\end{proof}
\section{环,子环}
\begin{definition}[环]
    设$L$是一个非空集合,在$L$上定义了两个代数运算,一个叫加法,记为$a+b$,一个叫乘法,记为$ab$.若这两种运算具有性质\begin{align*}
        &(1)\text{$L$对于加法构成Abel群};\\
        &(2)\text{$L$对于乘法满足结合律};\\
        &(3)\text{$L$满足乘法对加法的分配律},
    \end{align*}则称$L$为环.
\end{definition}
\begin{definition}[子环]
    设$S$是环$L$的非空子集合,若$S$对于$L$的两种运算也成环,则称环$S$是环$L$的子环.
\end{definition}
\begin{proposition}
    环$L$的非空子集合$S$成环的充分必要条件为$S$对于加法是子群且对于乘法封闭.
\end{proposition}
\begin{proof}
    必要性是显然的,下面证明充分性.

    (1)$S$对于加法构成Abel群:任取$a,b\in S$,于是$a,b\in L$,所以\begin{align*}
        ab\xlongequal[]{L\text{对于乘法构成Abel群}}ba,
    \end{align*}可见$S$关于加法构成的子群满足交换律,所以$S$为Abel群;

    (2)$S$对于乘法满足结合律:任取$a,b,c\in S$,有$a,b,c\in L$,所以\begin{align*}
        a(bc)=(ab)c=abc\in S,
    \end{align*}可见$S$对于乘法满足结合律;

    (3)$S$满足乘法对于加法的分配律:任取$a,b,c\in S$,有$a,b,c\in L$,所以\begin{align*}
        a(b+c)\xlongequal[]{L\text{满足乘法对加法的分配律}}ab+ac\in S,
    \end{align*}可见$S$满足乘法对于加法的分配律.
\end{proof}
\begin{definition}[同构映射]
    设$L$与$L'$是两个环,若有$L$到$L'$的一一对应$\sigma$满足如下性质\begin{align*}
        &(1)\sigma(a+b)=\sigma(a)+\sigma(b);\\
        &(2)\sigma(ab)=\sigma(a)\sigma(b),
    \end{align*}其中$a,b\in L$,则称$L$与$L'$同构,称具有以上性质的$\sigma$为一个同构映射(简称同构).
\end{definition}
\section{各种特殊类型的环}
\begin{proposition}\label{auhisgnjdvk}\ 
    \begin{center}
        \begin{tikzcd}
            &  & \text{环} \arrow[d, "\text{存在幺元}" description] \arrow[rrd, "\text{无零因子}" description] \arrow[lld, "\text{乘法可交换}" description] &                                             &                  \\
            \text{交换环} \arrow[rrd] \arrow[rrrddd] &  & \text{幺环} \arrow[rd]                                                                                    &                                             & \text{无零因子环} \arrow[ld] \\
            &  & \text{交换整环}                                                                                             & \text{整环} \arrow[d, "\text{非零元可逆}" description] \arrow[l] &                  \\
            &  &                                                                                                  & \text{体} \arrow[d]                                 &                  \\
            &  &                                                                                                  & \text{域}                                           &                 
        \end{tikzcd}
    \end{center}
\end{proposition}
\begin{remark}
    幺元:设$L$是环.若$e\in L$满足\begin{align*}
        \forall a\in L,ae=ea=a,
    \end{align*}则称$e$为环$L$的幺元(幺元),简记为$1$;

    用$0$表示环中加法群的幺元(即零元素);

    零因子:设$L$是环.若有$0\neq a\in L,0\neq b\in L$满足$ab=0$,则称$a$为一个左零因子,称$b$为一个右零因子.
\end{remark}
\begin{lemma}\label{zncxkjvna}
    非零元可逆$\rtc$无零因子.
\end{lemma}
\begin{proof}
    (1)非零元可逆$\Rightarrow$无零因子:

    设$L$是环且非零元可逆.假设$a\in L$是$L$的左零因子(右零因子同理),则有\begin{align*}
        ab=0\text{且}a\neq0,b\neq0.
    \end{align*}设$c\in L$是$a$的逆元,即\begin{align*}
        ac=ca=1,
    \end{align*}于是\begin{align*}
        &(ca)b=c(ab)\\
        &(ca)b=1b=b\neq0\\
        &c(ab)=c0=0,
    \end{align*}得到矛盾,从而$L$无零因子,问题得证.

    (2)无零因子$\nRightarrow$非零元可逆:

    如整数环.
\end{proof}
\begin{definition}[子域]
    若域$F$的子环$S$是域,则称$S$是域$F$的子域.
\end{definition}
\section{环的同态,理想}
\begin{definition}[同态]
    设$L,L'$是两个环,$\sigma$是$L$到$L'$的映射.若对$\forall a,b\in L,\sigma$具有性质\begin{align*}
        &(1)\sigma(a+b)=\sigma(a)+\sigma(b);\\
        &(2)\sigma(ab)=\sigma(a)\sigma(b),
    \end{align*}就称$\sigma$为环$L$到环$L'$的一个同态映射(简称同态),简记为$\sigma:L\to L'$.
\end{definition}
\begin{remark}
    (1)由同态的定义可以看出$\sigma(L)$是$L'$的子环;

    (2)若$\sigma(L)=\{0\}$,称$\sigma$为零同态;

    (3)若$\sigma(L)=L'$,称$\sigma$为满同态,称$L'$为$L$的同态象.
\end{remark}
\begin{definition}[理想]
    设$L$成环,$I\subset L$为$L$的一个加法子群.若$\forall r\in L,\forall a\in L$,都有\begin{align*}
        ra\in I,ar\in I,
    \end{align*}就称$I$是$L$的理想(或双边理想).若只满足$ra\in I$(\text{或}$ar\in I$),则称$I$是$L$的右(或左)理想.
\end{definition}
\begin{remark}
    显然$\{0\}$与$L$都是$L$的理想,称它们为平凡的理想.
\end{remark}
\section{商环}
\begin{definition}[陪集]\label{pj}
    设环$I$是环$L$的理想,$I$作为$L$的加法群的子群,按如下方式定义陪集\begin{align*}
        &r+I(\forall r\in L)\text{为左陪集};&&I+r(\forall r\in L)\text{为右陪集},
    \end{align*}按如下方式定义陪集的加法与乘法\begin{align*}
        &(r_1+I)+(r_2+I)=r_1+r_2+I&&(\forall r_1,r_2\in L);\\
        &(r_1+I)(r_2+I)=r_1r_2+I&&(\forall r_1,r_2\in L),
    \end{align*}全体陪集所成的集合在这样规定的运算下成环.
\end{definition}
\begin{definition}[商环]
    设环$I$是环$L$的理想.$L$对于$I$的陪集在定义\ref{pj}的运算下所成的环称为$L$对于$I$的商环,记作$L/I$.
\end{definition}
设环$I$是环$L$的理想.不难发现$\sigma(a)=a+I,a\in L$是环$L$到商环$L/I$的满同态,且该同态的核为理想$I$.可见每个理想都是某一同态的核.
\begin{lemma}
    设$\sigma:L\to L'$,则$\ker{\sigma}$是$L$的理想.
\end{lemma}
\begin{proof}
    对$\forall a\in \ker\sigma,\forall b\in L$,有\begin{align*}
        \sigma(ab)\xlongequal[]{\sigma\text{是同态}}\sigma(a)\sigma(b)=0\sigma(b)=0\Lra ab\in\ker\sigma;\\
        \sigma(ba)\xlongequal[]{\sigma\text{是同态}}\sigma(b)\sigma(a)=\sigma(b)0=0\Lra ba\in\ker\sigma,
    \end{align*}可见$\ker\sigma$是$L$的理想.
\end{proof}
\begin{theorem}[环同态基本定理]\label{httjbdl}
    若$\sigma:L\to L'$,则$L/\ker{\sigma}\cong\im\sigma$.进一步,若$\sigma$是满同态,则$L/\ker{\sigma}\cong L'$.
\end{theorem}
\begin{proof}
    设$\varphi:L\to L/\ker{\sigma}$是自然同态,则得到两个满同态$\sigma$和$\varphi$,交换图如下:\begin{center}
        \begin{tikzcd}
            L \arrow[rr, "\sigma"] \arrow[d, "\varphi"'] &  & \im{\sigma} \\
            L/\ker{\sigma} \arrow[rru, "\psi"', dashed]             &  &   
            \end{tikzcd}
    \end{center}其中虚线部分的$\psi$表示我们要找的同构.定义映射\begin{align*}
        \psi_0(\ker\sigma+a)=\sigma(a),
    \end{align*}显然$\psi_0$是良定义的.

    对$\forall a,b\in L$,有\begin{align*}
        \psi_0[(\ker\sigma+a)+(\ker\sigma+b)]\xlongequal[]{\ker\sigma\text{是理想}}&\psi_0(\ker\sigma+a+b)\\
        =&\sigma(a+b)\\
        \xlongequal[]{\sigma\text{是同态}}&\sigma(a)+\sigma(b)\\
        =&\psi_0(\ker\sigma+a)+\psi_0(\ker\sigma+b),
    \end{align*}由此可见$\psi_0$保持加法;\begin{align*}
        \psi_0[(\ker\sigma+a)(\ker\sigma+b)]\xlongequal[]{\ker\sigma\text{是理想}}&\psi_0(\ker+ab)\\
        =&\sigma(ab)\\
        \xlongequal[]{\sigma\text{是同态}}&\sigma(a)\sigma(b)\\
        =&\psi_0(\ker\sigma+a)\psi_0(\ker\sigma+b),
    \end{align*}由此可见$\psi_0$保持乘法,于是$\psi_0:L/\ker\sigma\to\im\sigma$.

    对$\forall\sigma(a)=\sigma(b)$,有\begin{align*}
        &\sigma(a)-\sigma(b)=0\\
        \xLra[]{\sigma\text{是同态}}&\sigma(a-b)=0\\
        \Lra&a-b\in\ker\sigma\\
        \xLra[]{\text{引理}\ref{jskdf}}&\ker\sigma+a=\ker\sigma+b\\
        \Lra&\psi_0\text{是单射}.
    \end{align*}

    $\psi_0$显然是满射.

    综上所述,$\psi_0$是同构映射.取$\psi=\psi_0$即证明了$L/\ker\sigma\cong\im\sigma$.

    进一步,若$\sigma$是满同态,则$L'\cong\im\sigma$,从而$L/\ker I\cong L'$.
\end{proof}
\section{特征}
设$F$是域,$e$是$F$中的幺元.若$e$是有限阶元素,即存在正整数$m$使得$me=0$,则将$m$定义为$F$的幺元在$F$的加法群中的阶.显然$m$一定是素数.

\begin{definition}[特征]
    设$F$是域.若$F$的幺元$e$在$F$的加法群中是有限阶元素,阶为$p$,就称域$F$的特征为$p$l若幺元是无限阶元素,就称域$F$的特征为$0$.域$F$的特征记为$\chi(F)$.
\end{definition}
\begin{proposition}
    在域的加法群中,任一非零元素都与幺元有相同的阶.
\end{proposition}
\begin{proof}
    设$a$是域$F$的任一非零元,由\begin{align*}
        ma=mae\xlongequal[]{ae\text{乘法可交换}}a(me)
    \end{align*}可知,$ma=0$当且仅当$me=0$,问题得证.
\end{proof}
\begin{theorem}
    设$F$为域.若$\chi(F)=p\neq0,$则$F$包含与$Z/pZ$同构的子域;若$\chi(F)=0,$则$F$包含与有理数域同构的子域.
\end{theorem}
\begin{proof}
    首先按如下方式定义整数环到$F$的映射$\sigma$\begin{align*}
        \sigma(n)=ne.
    \end{align*}注意到$\forall n,m\in\Z$\begin{align*}
        &\sigma(n+m)=(n+m)e=ne+me;\\
        &\sigma(nm)=(nm)e=(ne)(me)=\sigma(n)\sigma(m),
    \end{align*}于是$\sigma:\Z\to F$.

    (1)若$\chi(F)=p\neq0$,令$\sigma(n)=0$,有\begin{align*}
        0=\sigma(n)=ne\iff n\in p\Z,
    \end{align*}可见$\ker{\sigma}=p\Z$.易见\begin{align*}
        \im\sigma=\l\{e,2e,\cdots,(p-1)e,0\r\}\subset F,
    \end{align*}不难验证$\im\sigma$构成域$F$的子域.根据环同态基本定理(定理\ref{httjbdl}),有\begin{align*}
        F/p\Z\cong\im\sigma.
    \end{align*}

    (2)若$\chi(F)=0$,令$\sigma(n)=e$可推出$n=0$,即$\ker\sigma=\l\{0\r\}$,所以$\sigma$是单射.易见\begin{align*}
        \im\sigma=\l\{ne\r|n\in\Z\}
    \end{align*}与整数环$\Z$同构.按如下方式扩充$\sigma$的定义\begin{align*}
        \sigma\l(\frac{m}{n}\r)=(ne)^{-1}(me),
    \end{align*}由于当$\frac{m}{n}=\frac{m'}{n'}$时\begin{align*}
        (ne)^{-1}(me)^{-1}=(n'e)^{-1}(m'e)^{-1}\iff\sigma\l(\frac{m}{n}\r)=\sigma\l(\frac{m'}{n}\r),
    \end{align*}所以这是良定义的.易见\begin{align*}
        \im\sigma=\l\{(ne)^{-1}me:n\in\Z\text{且}n\neq0;m\in\Z\r\}\subset F
    \end{align*}构成$F$的子域,而$\im\sigma\cong\Q$,亦即$F$有一个同构于有理数域$\Q$的子域.

    综上所述,问题得证.
\end{proof}
    \section{CHAPTER1习题}
\begin{problem}[P54T7]
    设$G$是群,$a,b\in G$.若$a^{-1}ba=b^r(r\in \N_+)$,证明$a^{-i}ba^i=b^{r^i}(1,2,\cdots)$.
\end{problem}
\begin{proof}
    使用数学归纳法.

    (1)题设条件已经说明,当$i=1$时结论成立;

    (2)假设当$i=n$时结论成立即$a^{-n}ba^n=b^{r^n}$,于是\begin{align*}
        a^{-(n+1)}ba^{n+1}=&a^{-1}(a^{-n}ba^n)a\\
        =&a^{-1}b^{r^n}a\\
        =&(a^{-1}ba)^{r^n}\\
        =&(b^r)^{r^n}\\
        =&b^{r^{n+1}},
    \end{align*}可见当$i=n+1$时结论也成立.
    
    综上所述,问题得证.
\end{proof}
\begin{problem}[P54T8]
    证明:群$G$为交换群$\iff$映射$x\mapsto x^{-1}$为同构映射.
\end{problem}
\begin{proof}
    设$\varphi:G\to G',x\mapsto x^{-1}$,不难发现$G=G'$.

    (1)必要性:

    令$\varphi(x)=e$,有$x^{-1}=e\Lra x=e\Lra\ker\varphi=\l\{e\r\}$,可见$\varphi$是单射;

    $\forall x\in G'=G$,$\exists x^{-1}\in G$满足$\varphi(x^{-1})=x$,可见$\varphi$是满射;

    $\forall x,y\in G$,有\begin{align*}
        \varphi(xy)=(xy)^{-1}=&y^{-1}x^{-1}\\
        \xlongequal[]{G\text{是交换群}}&x^{-1}y^{-1}\\
        =&\varphi(x)\varphi(y),
    \end{align*}于是$\varphi$是同态.

    必要性得证.

    (2)充分性:

    $\forall x,y\in G$,有$x^{-1},y^{-1}\in G$,并且\begin{align*}
        &\varphi(x^{-1}y^{-1})\xlongequal[]{\varphi\text{是同态}}\varphi(x^{-1})\varphi(y^{-1})\\
        \Lra&yx=xy\\
        \Lra&G\text{是交换群}.
    \end{align*}
    
    充分性得证.

    综上所述,问题得证.
\end{proof}
\begin{problem}[P54T9]
    设$S$为群$G$的非空子集合,在$G$中定义关系$a\sim b$当且仅当$ab^{-1}\in S$.证明这是等价关系的充要条件为$S$为$G$的子群.
\end{problem}
\begin{proof}
    先给出等价关系的定义.
    
    称满足如下三条性质的关系$\sim$为等价关系\begin{align*}
        \text{(i)}&\text{反身性:$a\sim a$;}\\
        \text{(ii)}&\text{对称性:若$a\sim b$,则$b\sim a$;}\\
        \text{(iii)}&\text{传递性:若$a\sim b,b\sim c$,则$a\sim c$.}
    \end{align*}

    (1)必要性:

    对$\forall a,b\in S$亦即$ae^{-1},be^{-1}\in S$,有\begin{align*}
        ae^{-1}(be^{-1})^{-1}\in S,
    \end{align*}即$ab^{-1}\in S$,可见$S<G$.
    
    必要性得证.

    (2)充分性:

    由于$S$非空,所以$\forall s\in S$,有\begin{align*}
        ss^{-1}\in S,
    \end{align*}即$s\sim s$,反身性得证;

    任取$a\in S$,由于$S$成群,所以$a^{-1}\in S$,进而若$a\sim b$即$ab^{-1}\in S$即$a\sim b$,有\begin{align*}
    ba^{-1}=(ab^{-1})^{-1}\in S,
    \end{align*}即$b\sim a$,对称性得证;

    设$a\sim b,b\sim c$即$ab^{-1}\in S,bc^{-1}\in S$,由$S$成群可知\begin{align*}
    (ab^{-1})(bc^{-1})\in S,
    \end{align*}即$a\sim c$,传递性得证.

    充分性得证.

    综上所述,问题证毕.
\end{proof}
\begin{problem}[P55T20]
    设群$H,K$为群$G$的子群,证明$HK$为$G$的子群当且仅当$HK=KH$.
\end{problem}
\begin{proof}
    (1)必要性

    按以下方式定义从$HK$到$HK$的一一对应$\varphi_1$\begin{align*}
        \varphi_1(hk)=(hk)^{-1},\forall hk\in HK.
    \end{align*}注意到$\im\varphi_1=HK$,并且\begin{align*}
        (hk)^{-1}=k^{-1}h^{-1}\in KH,
    \end{align*}即$HK=\im\varphi_1\subset KH$.同理,按以下方式定义$KH$到$KH$的一一对应$\varphi_2$可证$KH\subset HK$\begin{align*}
        \varphi_2(kh)=k^{-1}h^{-1},\forall kh\in KH.
    \end{align*}由$HK\subset KH$及$KH\subset HK$可得$HK=KH$,必要性得证.

    (2)充分性

    对任意$h_1k_1,h_2k_2\in HK$,有\begin{align*}
        h_1k_1(h_2k_2)^{-1}=&h_1k_1k_2^{-1}h_2^{-1}\\
        =&h_1(k_1k_2^{-1}h_2^{-1}),
    \end{align*}而\begin{align*}
        k_1k_2^{-1}h_2^{-1}\in KH=HK,
    \end{align*}所以$h_1k_1(h_2k_2)^{-1}\in HK$,亦即\begin{align*}
        \forall a,b\in HK\Lra ab^{-1}\in HK,
    \end{align*}可见$HK<G$,充分性得证.

    综上所述,问题证毕.
\end{proof}
\begin{problem}[P56T28]
    在整数集$\Z$上重新定义加法与乘法为\begin{align*}
        &a\oplus b=ab,&&a\odot b=a+b.
    \end{align*}试问$\Z$在新定义的运算下是否成环.
\end{problem}
\begin{solution}
    不能成环,理由如下.

    假设$\Z$在新定义的运算下成环,则$\Z$关于加法成交换群.对$\forall n\in\Z$
    \begin{align*}
        1\oplus n=1\cdot n=n,
    \end{align*}所以$\Z$在新定义的运算下,关于加法的所成的交换群中的幺元是$1$.注意到$\forall m\in\Z$\begin{align*}
        0\oplus m=0\cdot m=0\neq1,
    \end{align*}所以在此加法群中,$0$无逆元,这与$\Z$关于加法成交换群矛盾,所以$\Z$在新定义的运算下不成环.
\end{solution}
\begin{problem}[P56T29]
    设$L$为有幺元的交换环,在$L$中定义\begin{align*}
        &a\oplus b=a+b-1,\\
        &a\odot b=a+b-ab.
    \end{align*}证明在新定义的运算下,$L$仍为有幺元的交换环,并且与原来的环同构.
\end{problem}
\begin{proof}
    (1)对任意$a,b,c\in L$\begin{align*}
        &(a\oplus b)\oplus c\\
        =&(a+b-1)\oplus c\\
        =&(a+b-1)+c-1\\
        =&a+b+c-2\\
        =&a+(b+c-1)-1\\
        =&a+(b\oplus c)-1\\
        =&a\oplus(b\oplus c),
    \end{align*}$L$关于$\oplus$满足结合律;

    (2)对任意$a\in L$\begin{align*}
        &1\oplus a\\
        =&1+a-1\\
        =&a,
    \end{align*}$L$关于$\oplus$有幺元;

    (3)对任意$a\in L$\begin{align*}
        &(-a)\oplus a\\
        =&-a+a-1\\
        =&1,
    \end{align*}$L$中的元素关于$\oplus$有逆元;

    (4)对任意$a,b\in L$\begin{align*}
        &a\oplus b\\
        =&a+b-1\\
        =&b+a-1\\
        =&b\oplus a,
    \end{align*}$L$关于$\oplus$可交换;

    (5)对任意$a,b,c\in L$\begin{align*}
        &(a\oplus b)\odot c\\
        =&(a+b-1)\odot c\\
        =&(a+b-1)+c-(a+b-1)c\\
        =&a+b+2c-ac-bc-1,
    \end{align*}\begin{align*}
        &(a\odot c)\oplus(b\odot c)\\
        =&(a\odot c)+(b\odot c)-1\\
        =&(a+c-ac)+(b+c-bc)-1\\
        =&a+b+2c-ac-bc-1,
    \end{align*}$L$满足$\odot$对于$\oplus$的分配律;

    (6)对任意$a,b\in L$\begin{align*}
        &a\odot b\\
        =&a+b-ab\\
        =&b+a-ba\\
        =&b\odot a,
    \end{align*}$L$关于$\odot$满足交换律;

    (7)对任意$a\in L$,存在$0\in L$满足\begin{align*}
        &0\odot a\\
        =&0+a-0a\\
        =&a,
    \end{align*}$L$关于$\odot$有幺元.

    (1)$\sim$(7)说明$L$成有幺元的交换环,其中零元为$1$,幺元为$0$.

    定义$\varphi$为$(L;+,\cdot)\to(L;\oplus,\odot)$的映射\begin{align*}
        \varphi(x)=1-x,
    \end{align*}显然$\varphi$为双射.

    注意到\begin{align*}
        \varphi(x+y)=&1-x-y,\\
        \varphi(x)\oplus\varphi(y)=&(1-x)\oplus(1-y)\\
        =&(1-x)+(1-y)-1=1-x-y,
    \end{align*}即$\varphi(x+y)=\varphi(x)\oplus\varphi(y)$;\begin{align*}
        \varphi(xy)=&1-xy,\\
        \varphi(x)\odot\varphi(y)=&(1-x)\odot(1-y)\\
        =&(1-x)+(1-y)-(1-x)(1-y)\\
        =&2-x-y-(1-y-x+xy)\\
        =&1-xy,
    \end{align*}即$\varphi(xy)=\varphi(x)\odot\varphi(y)$.可见$\varphi$是同态映射.

    综上所述,$\varphi:(L;+,\cdot)\to(L;\oplus,\odot)$为同构映射,问题得证.
\end{proof}
\begin{problem}[P56T30]
    给环出$L$与它的子环$S$的例子,它们分别具有下列性质

    (1)$L$有幺元,$S$无幺元;

    (2)$L$无幺元,$S$有幺元;

    (3)$L,S$均有幺元,但不相同;

    (4)$L$不交换,$S$交换.
\end{problem}
\begin{solution}
    (1)$L=(\Z;+,\cdot),S=(2\Z;+,\cdot)$.\\(1.1)对于$L$:

    (1.1.1)$\forall a,b,c\in \Z$\begin{align*}
        (a+b)+c=a+b+c=a+(b+c),
    \end{align*}可见$(\Z;+)$满足结合律;

    (1.1.2)$\forall a\in\Z,\exists 0\in\Z$满足\begin{align*}
        0+a=a,
    \end{align*}可见$(\Z;+)$存在左幺元;

    (1.1.3)$\forall a\in\Z,\exists-a\in\Z$满足\begin{align*}
        -a+a=0,
    \end{align*}可见$(\Z;+)$中的任意元素都有左逆元;

    (1.1.4)$\forall a,b\in\Z$,有\begin{align*}
        a+b=b+a\in\Z,
    \end{align*}可见$(\Z;+)$满足交换律;

    (1.1.5)$\forall a,b,c\in\Z$,有\begin{align*}
        a(b+c)=ab+ac,
    \end{align*}可见$(\Z;+,\cdot)$满足乘法对于加法的分配律.\\(1.1.1)$\sim$(1.1.5)说明$(\Z;+,\cdot)$成环.注意到$\forall a\in\Z$,有$1\in \Z$满足\begin{align*}
        1\cdot a=a,
    \end{align*}所以$(\Z;+,\cdot)$有幺元$1$.\\(1.2)对于$S$:

    同理可证$S$成环.假设$S$有幺元$e$,则$\forall s\in S$\begin{align*}
        es=s,
    \end{align*}现取$n\in\Z$且$m\neq0$,则$2n\in2\Z$且\begin{align*}
        e(2n)=2n\xLra[]{\text{等式两端同时除以$2n$}}e=1\notin2\Z,
    \end{align*}这与$e\in S$矛盾,所以$S$没有幺元.

    (2)\begin{align*}
        L=\l(\l\{\l(\begin{matrix}
            a&b\\
            0&0
        \end{matrix}\r):a,b\in\R\r\};+,\cdot\r),S=\l(\l\{\l(\begin{matrix}
            a&0\\
            0&0
        \end{matrix}\r):a\in\R\r\};+\cdot\r).
    \end{align*}(2.1)对于$L$:

    (2.1)容易验证$L$成环.令$e=\l(\begin{matrix}
        x_1&x_2\\
        x_3&x_4
    \end{matrix}\r)$满足\begin{align*}
        &\l\{\begin{matrix}
            e\l(\begin{matrix}
                a&b\\
                0&0
            \end{matrix}\r)=\l(\begin{matrix}
                x_1&x_2\\
                x_3&x_4
            \end{matrix}\r)\l(\begin{matrix}
                a&b\\
                0&0
            \end{matrix}\r)=\l(\begin{matrix}
                a&b\\
                0&0
            \end{matrix}\r)\\
            \l(\begin{matrix}
                a&b\\
                0&0
            \end{matrix}\r)e=\l(\begin{matrix}
                a&b\\
                0&0
            \end{matrix}\r)\l(\begin{matrix}
                x_1&x_2\\
                x_3&x_4
            \end{matrix}\r)=\l(\begin{matrix}
                a&b\\
                0&0
            \end{matrix}\r)
        \end{matrix}\r.\text{对任意$a,b\in\R$均成立}\\
        \iff&\begin{cases}
            ax_1=a\\
            bx_1=b\\
            ax_3=0\\
            bx_4=0\\
            ax_1+bx_3=a\\
            ax_2+bx_4=b\\
            ax_3=0\\
            bx_3=0
        \end{cases}\text{对任意$a,b\in\R$均成立},
    \end{align*}解之可得\begin{align*}
        e=\l(\begin{matrix}
            1&0\\
            0&1
        \end{matrix}\r)\notin L,
    \end{align*}亦即$L$没有幺元.

    容易验证$S$成环.注意到$\forall s=\l(\begin{matrix}
        a&0\\
        0&0
    \end{matrix}\r)\in S,\exists e=\l(\begin{matrix}
        1&0\\
        0&0
    \end{matrix}\r)\in S$满足\begin{align*}
        es=se=s,
    \end{align*}所以$S$有幺元$\l(\begin{matrix}
        1&0\\
        0&0
    \end{matrix}\r)$.

    (3)\begin{align*}
        L=\l(\l\{\l(\begin{matrix}
            a&0\\
            0&b
        \end{matrix}\r):a,b\in\R\r\};+,\cdot\r),S=\l(\l\{\l(\begin{matrix}
            a&0\\
            0&0
        \end{matrix}\r):a\in\R\r\};+,\cdot\r).
    \end{align*}取$e_1=\l(\begin{matrix}
        1&0\\
        0&1
    \end{matrix}\r),e_2=\l(\begin{matrix}
        1&0\\
        0&0
    \end{matrix}\r)$验证它们分别是$L$与$S$中的幺元即可.

    (4)\begin{align*}
        L=\l(\l\{\l(\begin{matrix}
            a&0\\
            b&0
        \end{matrix}\r):a,b\in\R\r\};+,\cdot\r),S=\l(\l\{\l(\begin{matrix}
            a&0\\
            0&0
        \end{matrix}\r):a\in\R\r\};+,\cdot\r).
    \end{align*}(4.1)对于$L$:
    
    令$l_1=\l(\begin{matrix}
        1&0\\
        2&0
    \end{matrix}\r),l_2=\l(\begin{matrix}
        3&0\\
        4&0
    \end{matrix}\r)\in L$,易见\begin{align*}
        l_1l_2=\l(\begin{matrix}
            3&0\\
            6&0
        \end{matrix}\r)\neq\l(\begin{matrix}
            3&0\\
            4&0
        \end{matrix}\r)=l_2l_1,
    \end{align*}于是$L$不交换.\\(4.2)对于$S$:

    任取$\l(\begin{matrix}
        a&0\\
        0&0
    \end{matrix}\r),\l(\begin{matrix}
        b&0\\
        0&0
    \end{matrix}\r)\in S$,注意到\begin{align*}
        &\l(\begin{matrix}
            a&0\\
            0&0
        \end{matrix}\r)\l(\begin{matrix}
            b&0\\
            0&0
        \end{matrix}\r)\\
        =&\l(\begin{matrix}
            ab&0\\
            0&0
        \end{matrix}\r)\\
        =&\l(\begin{matrix}
            ba&0\\
            0&0
        \end{matrix}\r)\\
        =&\l(\begin{matrix}
            b&0\\
            0&0
        \end{matrix}\r)\l(\begin{matrix}
            a&0\\
            0&0
        \end{matrix}\r),
    \end{align*}所以$S$交换.
\end{solution}
\begin{problem}[P56T31]
    环$L$中元素$e_L$称为左幺元,若对$\forall a\in L$\begin{align*}
        e_La=a;
    \end{align*}元素$e_R$称为右幺元,若对$\forall a\in L$\begin{align*}
        e_Ra=a;
    \end{align*}证明

    (1)若$L$既有左单位又有右单位,则$L$有幺元;

    (2)若$L$有左单位,无零因子,则$L$有幺元;

    (3)若$L$有左单位,无右单位,则$L$至少有两个左单位.
\end{problem}
\begin{proof}
    (1)由\begin{align*}
        e_Le_R=\l\{\begin{matrix}
            e_R\text{($e_L$是左幺元)};\\
            e_L\text{($e_R$是右幺元)}
        \end{matrix}\r.
    \end{align*}可知$e_L=e_R$,所以$L$有幺元;

    (2)在等式$e_La=a$两端同时左乘$a$可得\begin{align*}
        &ae_La=a^2\\
        \Lra&(ae_L-a)a=0\\
        \Lra&ae_L-a=0\\
        \Lra& ae_L=a,
    \end{align*}可见$L$有幺元;

    (3)设$e_L$为$L$的一个左单位,由于$L$无右单位,所以$\exists x\in L$,满足\begin{align*}
        &xe_L\neq x\\
        \Lra&xe_L-x+e_L\neq e_L.
    \end{align*}注意到$\forall a\in L$\begin{align*}
        (xe_L-x+e_L)a=a,
    \end{align*}所以$xe_L-x+e_L$是异于$e_L$的左单位,所以$L$至少有两个左单位.
\end{proof}
\begin{problem}[P56T32]
    设$F$为域.证明$F$无非平凡的理想.
\end{problem}
\begin{proof}
    设$I\neq\varnothing$是$F$的理想.首先证明$I$一定含有零元.

    对$\forall a\in I$,若$a$是零元,则显然$I$含有零元;若$a$不是零元,由\begin{align*}
        a\in I\subset F
    \end{align*}可知$a$在$F$中存在逆元$-a$,于是由理想的定义可知\begin{align*}
        0=a+(-a)\in I,
    \end{align*}所以$I$有零元.

    由理想的定义有$\forall f\in F$\begin{align*}
        f=0+f\in I,
    \end{align*}所以$F\subset I$,从而$F=I$.可见域$F$没有非平凡的理想.
\end{proof}
\begin{problem}[P57T35]
    设$L$为有幺元的交换环.若$L$无非平凡的理想,则$L$为域.
\end{problem}
\begin{proof}
    由命题\ref{auhisgnjdvk}与引理\ref{zncxkjvna}可知,若能证明$L$的非零元可逆,便能推出$L$为域.

    任取$0\neq a\in L$,令\begin{align*}
        La=\l\{la:l\in L\r\},
    \end{align*}首先证明$La$是$L$的加法子群.

    对$\forall l_1a,l_2a\in La$,有\begin{align*}
        l_1a-l_2a=(l_1-l_2)a,
    \end{align*}注意到$l_1-l_2\in L$,于是\begin{align*}
        (l_1-l_2)a\in La\Lra l_1a-l_2a\in La,
    \end{align*}所以$La$是$L$的加法子群.

    然后证明$La$是$L$的理想.对$\forall la\in La,b\in L$,注意到$bl,lb\in L,ab=ba$,所以\begin{align*}
        &lab=l(ab)=l(ba)=lba\in La\\
        &bla\in La,
    \end{align*}所以$La$是$L$的理想.因为$La\neq\varnothing$且$L$没有非平凡的理想,所以$La=L$.

    最后证明$L$即$La$是域.由于$1\in L=La$,所以\begin{align*}
        \exists b\in L\st ba=1,
    \end{align*}所以$a$有逆元$b$,亦即$L$的非零元均有逆元,从而$L$是域.问题证毕.
\end{proof}
    \input{群.tex}
    \section{CHAPTER2习题}
\begin{problem}[P97T1]
    已知$G$是有限群,$N\lhd G$,$\l(|N|,[G:N]\r)=1$.证明:若元素$a$的阶整除$|N|$,则$a\in N$.
\end{problem}
\begin{proof}
    考虑自然同态\begin{align*}
        \pi:G\to&G/N\\
        g\mapsto&gN,
    \end{align*}于是\begin{align*}
        \l[\pi(a)\r]^{o(a)}=&(aN)^{o(a)}\\
        \xlongequal[]{N\lhd G}&a^{o(a)}N^{o(a)}\\
        =&N,
    \end{align*}又$N$是$G/N$的幺元,所以\begin{align}\label{snjdkk}
        \l|\pi(a)\r|\Big|\l|o(a)\r|\Big|\l|N\r|.
    \end{align}

    注意到\begin{align*}
        aN\in G/N,
    \end{align*}所以\begin{align*}
        \l|aN\r|\Big|\l[G:N\r],
    \end{align*}即\begin{align}\label{asjgndk}
        \l|\pi(a)\r|\Big|\l[G:N\r].
    \end{align}

    由式\eqref{snjdkk}与\eqref{asjgndk}结合$\l(|N|,[G:N]\r)=1$可知\begin{align*}
        &|\pi(a)|=1\\
        \Lra&\pi(a)=N\\
        \Lra&aN=N\\
        \Lra&a\in N,
    \end{align*}问题得证.
\end{proof}
\begin{problem}[P97T2]
    设$c$是群$G$中阶为$rs$的元素,其中$(r,s)=1$.证明$c$可以表示成$c=ab$,其中$a$的阶为$r,b$的阶为$s$,且$a,b$都是$c$的方幂.
\end{problem}
\begin{proof}
    由$(r,s)=1$可知\begin{align*}
        \exists u,v\in\Z\st ur+vs=1,
    \end{align*}不难验证\begin{align*}
        &a=c^{vs}\\
        &b=c^{ur}
    \end{align*}满足题设要求.
\end{proof}
\begin{problem}[P97T3]
    已知群$G$中元素$a$的阶与正整数$k$互素,证明方程$x^k=a$在$<a>$内恰有一解.
\end{problem}
\begin{proof}
    由$\l(o(a),k\r)=1$可知\begin{align*}
        \exists u,v\in \Z\st uo(a)+vk=1,
    \end{align*}所以\begin{align*}
        a^{uo(a)+vk}=a,
    \end{align*}即\begin{align*}
        a^{vk}=a,
    \end{align*}由此可见$x=a^v$是$x^k=a$在$<a>$内的解.

    现设$x=a^v_0$也是$x^k=a$在$<a>$内的解,其中$0\leq v-v_0\leq o(a)-1$,于是\begin{align*}
        &a^{vk}=a\\
        &a^{v_0k}=a,
    \end{align*}即\begin{align*}
        vk\equiv1&\mod o(a)\\
        v_0k\equiv1&\mod o(a),
    \end{align*}所以\begin{align*}
    o(a)|(v-v_0)k,
    \end{align*}由此可见$v-v_0=0$即$v=v_0$.

    综上,问题证毕.
\end{proof}
\begin{problem}[P97T4]
    证明在群中,$ab$与$ba$有相同的阶.
\end{problem}
\begin{proof}
    注意到$ab=b^{-1}\cdot ba\cdot b$,所以\begin{align*}
        &ab=e\\
        \iff&b^{-1}\cdot ba\cdot b=e\\
        \iff&ba=e,
    \end{align*}可见问题得证.
\end{proof}
\begin{problem}[P97T10]
    证明$S_n$中的任意一个置换能由$n-1$个对换$(12),(13),\cdots,(1n)$生成,也能由$n-1$个对换$(12),(23),\cdots,(n-1\ n)$生成.
\end{problem}
\begin{proof}
    由引理\ref{p65yl}可知$S_n$中的任意一个置换都可以拆成若干个对换$(ij)$的复合,注意到\begin{align*}
        (1i)(1j)(1i)=(ij),
    \end{align*}所以任意一个置换都能由$(n-1)$个对换\begin{align*}
        (12),(13),\cdots,(1n)
    \end{align*}生成,第一个论断证毕.

    注意到\begin{align*}
        (1i)(i\ i+1)(1i)=(1\ i+1),
    \end{align*}故可用归纳法证明\begin{align*}
        (12),(23),\cdots,(n-1\ n)
    \end{align*}能生成\begin{align*}
        (12)(13),\cdots,(1n),
    \end{align*}再结合第一个论断可知它能生成$S_n$中的任意一个置换,第二个论断证毕.
\end{proof}
\begin{problem}[P97T16]
    设$H_1,H_2$是群$G$的两个子群,证明$H_1\cap H_2$的任一左陪集是$H_1$的一个左陪集与$H_2$的一个左陪集的交.
\end{problem}
\begin{proof}
    若能证明对$\forall g\in G$都有$g(H_1\cap H_2)=gH_1\cap gH_2$,则问题得证.证明思路为两者互相包含.

    (1)对$\forall$给定的$g\in G$,在$g(H_1\cap H_2)$中任取$gh_0$,有\begin{align*}
        &h_0\in H_1\cap H_2\\
        \Lra&gh_0\in gH_1,gh_0\in gH_2\\
        \Lra&gh_0\in gH_1\cap gH_2\\
        \Lra&g(H_1\cap H_2)\subset gH_1\cap gH_2.
    \end{align*}

    (2)反之,对任意给定的$g\in G$,在$gH_1\cap gH_2$中任取$gh_0$,有\begin{align*}
        &gh_0\in gH_1,gh_0\in gH_2\\
        \xLra[]{g\text{的任意性}}&h_0\in H_1,h_0\in H_2\\
        \Lra&h_0\in H_1\cap H_2\\
        \Lra&gh_0\in g(H_1\cap H_2)\\
        \Lra&gH_1\cap gH_2\subset g(H_1\cap H_2).
    \end{align*}

    由(1),(2)便知$g(H_1\cap H_2)=gH_1\cap gH_2$,综上,问题得证.
\end{proof}
\begin{problem}[P98T18]\label{p98t18}
    设$G$为有限群,$H<G$且$[G:H]=n>1$.证明$G$或者含有指数能整除$n!$的非平凡正规子群,或者$G$同构于$S_n$的一个子群.
\end{problem}
\begin{proof}
    设群$G$在子群$H$上的传递置换表示(定义\ref{idvhbu})为\begin{align*}
        \phi:G\to S_n.
    \end{align*}
    
    由$[G:H]>1$可知\begin{align}
        &\im\phi>1\nonumber\\
        \Lra&\im\phi\neq\l\{e\r\}\nonumber\\
        \Lra&\ker\phi\neq G\label{jnakd}.
    \end{align}

    (1)若$\ker\phi\neq\l\{e\r\}$,结合\eqref{jnakd}可知\begin{align}
        \text{$\ker\phi$是$G$的非平凡正规子群.}\label{uhnj}
    \end{align}注意到\begin{align}
        |\ker\phi|\Big|[G:H]\xlongequal[]{\text{群同态基本定理(定理\ref{qttjbdy})}}|\im\phi|\Big||S_n|=n!,\label{sjda}
    \end{align}所以由\eqref{uhnj}与\eqref{sjda}可知$\ker\phi$是$G$的指数能够整除$n!$的非平凡的正规子群.
    
    (2)若$\ker\phi=\l\{e\r\}$,则\begin{align*}
        &G/\ker\phi\cong\im\phi\\
        \iff&G\cong\im\phi<S_n,
    \end{align*}即$G$同构于$S_n$的子群$\im\phi$.

    由(1),(2)便知问题证毕.
\end{proof}
\begin{problem}[P98T19]
    设$G$为有限群,$p$是$|G|$的最小素因子.证明指数(定义\ref{vs})为$p$的子群(若存在)必正规.
\end{problem}
\begin{proof}
    设$H$是群$G$的指数为$p$的子群,即$[G:H]=p$,$\phi$是$G$在$H$上的传递置换表示(定义\ref{idvhbu}).

    (1)若$\ker\phi=\l\{e\r\}$,则由群同态基本定理(定理\ref{qttjbdy})可知\begin{align*}
        &G/\ker\phi\cong\im\phi\\
        \iff&G\cong\im\phi<S_p,
    \end{align*}于是\begin{align*}
        |G|=|\im\phi|\Big||S_p|=p!,
    \end{align*}结合$p$是$|G|$的最小素因子便知$|G|=p$,于是\begin{align*}
        &|H|=\frac{|G|}{[G:H]}=\frac{p}{p}=1\\
        \iff&H=\l\{e\r\}\\
        \Lra\ &H\text{是$G$的正规子群}.
    \end{align*}

    (2)若$\ker\phi\neq\l\{e\r\}$,则\begin{align*}
        &g\in\ker\phi\\
        \iff&\forall a_i\in G,ga_iH=a_iH\\
        \xLra[]{\text{当$a_i=e$时}}\ &gH=H\\
        \iff&g\in H,
    \end{align*}由此可见$\ker\phi\subset H$,从而$\ker\phi\lhd H$.注意到\begin{align*}
        &[G:\ker\phi]=|\im\phi|\Big||S_p|=p!\\
        \xLra[]{\text{$p$是$|G|$的最小素因子}}&[G:\ker\phi]=p\\
        \xLra[]{\ker\phi\lhd H,[G:H]=p}&\text{$H=\ker\phi$是$G$的指数为$p$的正规子群}.
    \end{align*}

    由(1),(2)可知问题证毕.
\end{proof}
\begin{problem}[P98T21]
    证明任一非交换的$6$阶群同构于$S_3$.
\end{problem}
证明该问题需要用到以下引理.
\begin{lemma}\label{sdfmk}
    素数阶群都是循环群.
\end{lemma}
\begin{proof}
    任取阶为素数$p$的有限群$G,e\neq g\in G$,于是由Lagrange定理(推论\ref{Lagrangedl})可知\begin{align*}
        &\exists k\in\N_+\st|G|=k|<g>|\\
        \xLra[]{\text{$p$是素数,$g\neq e$}}&|G|=|<g>|\\
        \Lra&G=<g>,
    \end{align*}亦即$G$是循环群,由$G$的任意性可知问题得证.
\end{proof}
\begin{proof}
    设$G$是非交换的$6$阶群.注意到\begin{align*}
        6=2\times3,
    \end{align*}所以由$Sl$第一定理(定理\ref{xldydl})可知$G$有阶为$2$的\Sl2$A$,阶为$3$的\Sl3$B$,由引理\ref{sdfmk}可知$A,B$都是循环群,故可设\begin{align*}
        \text{$A=<\alpha>,B=<\beta>$,其中$o(a)=2,o(b)=3$.}
    \end{align*}
    由于$G$不交换,故可用反证法证明\begin{align*}
        &\text{$\alpha,\beta$不交换}\\
        \Lra&\alpha\neq\beta,\alpha\neq\beta^2\\
        \xLra[]{\beta\cdot\beta^2=\beta^3=e}&\alpha\neq\beta^{-1}\\
        \Lra&\alpha\notin B,
    \end{align*}所以$B$的左陪集$B,\alpha B$满足\begin{align*}
        &B\cap\alpha B=\emptyset,|B|+|\alpha B|=|B|+|B|=3+3=6=|G|\\
        \Lra&G=B\cup\alpha B=\l\{e,\beta,\beta^2,\alpha,\alpha\beta,\alpha\beta^2\r\}.
    \end{align*}定义\begin{align*}
        \phi:G\to&S_6\\
        \alpha\mapsto&(12)\\
        \beta\mapsto&(123),
    \end{align*}利用上述讨论不难验证这是良定义的同构映射,从而\begin{align*}
        G\cong S_6.
    \end{align*}
\end{proof}
\begin{problem}[P98T29]
    已知$A,B$是有限群$G$的两个非空子集.若$|A|+|B|>|G|$,则$AB=G$.
\end{problem}
\begin{proof}
    反设$AB\neq G$,由于$AB\subset G$,所以\begin{align*}
        &G/AB\supset\l\{AB\r\}\\
        \Lra&[G:AB]>1\\
        \Lra&\exists g\in G\st g\notin AB\\
        \Lra&\exists g\in G\st\l\{g\r\}\cap AB=\emptyset\\
        \Lra&\exists g\in G\st gB^{-1}\cap A=\emptyset\\
        \xLra[]{|gB^{-1}|=|B^{-1}|=|B|}&\text{$G$中至少有$|B|$个元素不在$A$中}\\
        \xLra[]{|G|\neq\infty}&|G|-|A|\geq|B|,
    \end{align*}矛盾,假设不成立,从而$AB=G$.
\end{proof}
\begin{problem}[P98T33]
    已知$G$是有限群,$N\lhd G,P$为$N$的\Sl{p}.证明$G=N\cdot N_G(P)$.
\end{problem}
\begin{proof}
    对$P$在$G$中的任意一个共轭子群$Q$,存在$g\in G$使得\begin{align*}
        Q=gPg^{-1}\overset{\text{P<N}}{<}gNg^{-1}\xlongequal[]{N\lhd G}N,
    \end{align*}可见$P$在$G$的共轭子群都在$N$中,所以对$\forall g\in G,gPg^{-1}$是$N$的\Sl{p},并且

    (1)若$gPg^{-1}=P$,则\begin{align*}
        &g\in N_G(P)\subset N\cdot N_G(P)\\
        \Lra&G\subset N\cdot N_G(P).
    \end{align*}

    (2)若$gPg^{-1}\neq P$,注意到\begin{align*}
        &\text{$gPg^{-1}$是$P$在$G$中的共轭子群}\\
        &\text{$P$在$G$中的共轭子群也在$N$中},
    \end{align*}所以由$P$是$N$的\Sl{p}与推论\ref{p81tl1}可知\begin{align*}
        &\exists n\in N\st gPg^{-1}=nPn^{-1}\\
        \iff&(n^{-1}g)P(g^{-1}n)=P\\
        \iff&n^{-1}g\in N_G(P),
    \end{align*}记$t\in N_G(P)\st n^{-1}g=t$,就有\begin{align*}
        &g=nt\in N\cdot N_G(P)\\
        \xLra[]{g\text{的任意性}}&G\subset N\cdot N_G(P).
    \end{align*}

    由$N\cdot N_G(P)<G$与(1),(2)便知$G=N\cdot N_G(P)$.
\end{proof}
\begin{problem}[P98T35]
    已知$G$是有限群,$H<G,P$是$G$的\Sl{p}
\end{problem}
\begin{proof}
    \stars
\end{proof}
\end{document}