\chapter{群}
\section{群的同态定理}
\begin{theorem}[群的第一同构定理]\label{qddytgdl}
    设$G$是群,$H<G,N\lhd G$,则\begin{align*}
        &(1)HN<G\\
        &(2)H\cap N\lhd H\text{且}H/H\cap N\cong HN/N.
    \end{align*}
\end{theorem}
\begin{proof}
    (1)的证明

    $HN<G\iff$\begin{align*}
        (\mr{i})&HN\text{非空}\\
        (\mr{ii})&\forall a,b\in HN\Lra ab^{-1}\in HN.
    \end{align*}$HN$显然非空.对任意$h_1n_1,h_2n_2\in HN$注意到\begin{align*}
        &h_1n_1(h_2n_2)^{-1}\\
        =&h_1n_1n_2^{-1}h_2^{-1}\\
        =&h_1h_2^{-1}(h_2n_1n_2^{-1}h_2^{-1}).
    \end{align*}由于$N\lhd G$,所以$\forall g\in G,gNg^{-1}\in N$,又$h_2\in H\subset G$,所以\begin{align*}
        h_1h_2^{-1}(h_2n_1n_2^{-1}h_2^{-1})\in N,
    \end{align*}从而\begin{align*}
        h_1n_1(h_2n_2)^{-1}\in N,
    \end{align*}于是$HN<G$.

    (2)的证明

    定义映射\begin{align*}
        \varphi:H&\to HN/N\\
        h&\mapsto hN,
    \end{align*}显然这是良定义的.对$\forall h_1,h_2\in H$\begin{align*}
        \varphi(h_1h_2)&=h_1h_2N\\
        &=h_1Nh_2N\\
        &=\varphi(h_1)\varphi(h_2),
    \end{align*}可见$\varphi$是同态映射,所以由群同态基本定理(定理\ref{qttjbdy})\begin{align*}
        H/\ker\varphi\cong\im\varphi.
    \end{align*}
    
    注意到\begin{align*}
        &h\in\ker\varphi\\
        \iff&hN=N\\
        \iff&h\in N\\
        \overset{h\in H}{\iff}&h\in H\cap N,
    \end{align*}所以$\ker\varphi=H\cap N$,从而\begin{align*}
        &H\cap N\lhd H\\
        &H/H\cap N\cong\im\varphi.
    \end{align*}

    对$\forall hnN\in HN/N$,有\begin{align*}
        hnN=hN,
    \end{align*}所以$\exists h\in H\st$\begin{align*}
        \varphi(h)=hN,
    \end{align*}所以$\varphi$是满同态,从而\begin{align*}
        \im\varphi=HN/N,
    \end{align*}所以\begin{align*}
        H/H\cap N\cong HN/N.
    \end{align*}
\end{proof}
\begin{remark}
    类似可以证明:若$H<G,N\lhd G$,有\begin{align*}
        (1)&NH<G\\
        (2)&H\cap N\lhd H\text{且}H/H\cap N\cong NH/N.
    \end{align*}
\end{remark}
\begin{theorem}[群的第二同构定理]\label{qddetgdl}
    设$G$是群.若$N\lhd G,H\lhd G$并且$N\subset H$,则\begin{align*}
        (1)&H/N\lhd G/N\\
        (2)&(G/N)/(H/N)\cong G/H.
    \end{align*}
\end{theorem}
\begin{proof}
    (1)的证明

    定义映射\begin{align*}
        \varphi:G/N&\to G/H\\
        gN&\mapsto gH,
    \end{align*}显然$\varphi$是良定义的.对$\forall g_1,g_2\in G$\begin{align*}
        &\varphi(g_1Ng_2N)\\
        \xlongequal[]{N\lhd G}&\varphi(g_1g_2N)\\
        =&g_1g_2H\\
        \xlongequal[]{H\lhd G}&g_1Hg_2H\\
        =&\varphi(g_1N)\varphi(g_2N),
    \end{align*}可见$\varphi$是同态映射,所以由群同态基本定理(定理\ref{qttjbdy})\begin{align*}
        (G/N)/\ker\varphi\cong\im\varphi.
    \end{align*}注意到\begin{align*}
        &gN\in\ker\varphi\\
        \iff&gH=H\\
        \iff&g\in H\\
        \iff&gN\in H/N,
    \end{align*}可见$\ker\varphi=H/N$,所以\begin{align*}
        &H/N\lhd G/N\\
        &(G/N)/(H/N)\cong\im\varphi.
    \end{align*}

    (2)的证明
    
    对$\forall gH\in G/H,\exists g\in G\st$\begin{align*}
        \varphi(gN)=gH,
    \end{align*}可见$\varphi$是满同态,进而\begin{align*}
        \im\varphi=G/H,
    \end{align*}所以\begin{align*}
        (G/N)/(H/N)\cong G/H.
    \end{align*}
\end{proof}
\begin{remark}
    习惯上将群同态基本定理(定理\ref{qttjbdy}),群的第一同构定理(定理\ref{qddytgdl}),群的第二同构定理(定理\ref{qddetgdl})统称为群的同态定理.
\end{remark}
\begin{theorem}
    若$G$是群,$N\lhd G$,则$G$的所有包含$N$的子群与$G/N$的所有子群之间存在一一对应.
\end{theorem}
\begin{proof}
    取自然同态\begin{align*}
    \pi:G\to&G/N\\
        g\mapsto&gN,
    \end{align*}则问题转化为证明在集合$A$与$B$之间存在一一对应,其中\begin{align*}
        A&=\l\{H:H<G,H\supset\ker\pi\r\}\\
        B&=\l\{H':H'<\im\pi\r\}.
    \end{align*}

    设$H\in A,\pi_H$为映射$\pi$在子群$H$上的限制,容易验证$\im\pi_H\in B$.定义映射\begin{align*}
        \varphi:A&\to B\\
        H&\mapsto\im\pi_H,
    \end{align*}容易验证$\varphi$是良定义.若能证明$\varphi$是一一对应,则问题证毕.

    (1)$\varphi$是满射

    $\forall H'<\im\pi$即\begin{align*}
    H'=\l\{\pi(k):k\in K\subset G\r\}.
    \end{align*}注意到\begin{align*}
        &H'\text{成群}\\
        \iff&\forall\pi(k_1),\pi(k_2)\in H'\Lra\pi(k_1)[\pi(k_2)]^{-1}\in H'\\
        \iff&\forall\pi(k_1),\pi(k_2)\in H'\Lra\pi(k_1k_2^{-1})\in H'\\
        \iff&\forall k_1,k_2\in K\Lra k_1k_2^{-1}\in K\\
        \iff&K<G,
    \end{align*}所以\begin{align}\label{asijdufgn}
        H'=
        \l\{\pi(k):k\in K<G\r\}.
    \end{align}注意到\begin{align*}
        &\forall t\in\ker\pi\Lra\pi(t)=e\in H',
    \end{align*}因此不妨让式\eqref{asijdufgn}中的$K$包含$\ker\pi$,即\begin{align*}
        H'=&\l\{\pi(k):k\in K<G,K\supset\ker\pi\r\}\\
        =&\im\pi_K(K<G,K\supset\ker\pi),
    \end{align*}所以\begin{align*}
        B=&\l\{\im\pi_K:K<G,K\supset\ker\pi\r\}\\
        =&\l\{\im\pi_H:H<G,H\supset\ker\pi\r\}\\
        =&\l\{\im\pi_H:H\in A\r\},
    \end{align*}由此可见$\im\varphi=B$,即$\varphi$是满同态;

    (2)$\varphi$是单射\begin{align}
        \iff&\forall x_1,x_2\in A,\text{若}x_1\neq x_2,\text{则}\phi(x_1)\neq\phi(x_2)\nonumber\\
        \iff&\l|\l\{x\in A:\phi(x)=\phi(x_0),x_0\in A\r\}\r|=1\nonumber\\
        \iff&\l|\l\{H\supset N:\im\pi|_H=\im\pi|_{H_0},H_0\supset A\r\}\r|=1\nonumber\\
        \iff&\l\{H\supset N:\im\pi|_H=\im\pi|_{H_0},H_0\supset A\r\}=\l\{H_0\r\}\nonumber\\
        \iff&\text{若$H_0\supset N$,$H\supset N$且$\im\pi|_H=\im\pi|_{H_0}$,则$H=H_0$,}\label{asjdnfkaf}
    \end{align}注意到在\eqref{asjdnfkaf}的题设条件下,显然有$H\supset H_0$,故只需证明$H\subset H_0$便可得到$H=H_0$.

    在\eqref{asjdnfkaf}的题设条件下\begin{align*}
        &\forall h\in H,\exists h_0\in H_0\st\pi(h)=\pi(h_0)\\
        \iff&\forall h\in H,\exists h_0\in H_0\st hN=h_0N\\
        \iff&\forall h\in H,\exists h_0\in H_0\st h\in h_0N\subset H_0\\
        \Lra\ &H\subset H_0.
    \end{align*}所以$\phi$是单射.

    由(1),(2)便知\begin{align*}
        \varphi:A&\to B\\
        H&\mapsto\im\pi_H
    \end{align*}为一一对应,问题得证.
\end{proof}
\begin{definition}[由$S$生成的群]
    设$G$是群,$S$是$G$的非空子集合.$G$的包含$S$的最小的子群,称为由$S$生成的群,记作$<S>$,即\begin{align*}
        <S>=\bigcap_{S\subset H<G}H.
    \end{align*}
\end{definition}
\begin{proposition}
    设$(G,\cdot)$是群.若$S$是$G$的非空子集合,则\begin{align*}
        <S>=\l(\l\{\prod_{k=1}^{m}x_k:x_1,\cdots,x_m\in S\cup S^{-1},m\in\N_+\r\},\cdot\r),
    \end{align*}即集合$S\cup S^{-1}$中任意多个元素的乘积组成的集合关于$\cdot$构成$<S>$.
\end{proposition}
\begin{proof}
    容易验证群$G$的子集合$\l\{\prod_{k=1}^{m}x_k:x_1,\cdots,x_m\in S\cup S^{-1},m\in\N_+\r\}$关于群$G$的乘法成群,记此群为$A$.注意到$A$包含$S$,所以$A$包含:包含$S$的最小的子群,即$A\supset<S>$.

    反之,对$\forall a\in A$有\begin{align*}
        a=\l(\prod_{\alpha\in\Lambda_1}s_\alpha\r)\cdot\l(\prod_{\beta\in\Lambda_2}s_\beta\r),
    \end{align*}其中\begin{align*}
        &s_\alpha\in S\subset<S>,\alpha\in\Lambda_1\\
        &s_\beta\in S^{-1},\beta\in\Lambda_2,
    \end{align*}注意到$s_\beta^{-1}\in S\subset<S>,\beta\in\Lambda_2$,而$<S>$成群,从而\begin{align*}
        &s_\beta=\l(s_\beta^{-1}\r)^{-1}\in<S>,\beta\in\Lambda_2\\
        \Lra&a=\l(\prod_{\alpha\in\Lambda_1}s_\alpha\r)\cdot\l(\prod_{\beta\in\Lambda_2}s_\beta\r),
    \end{align*}可见$A\subset<S>$.

    综上,$A=<S>$,即\begin{align*}
        <S>=\l(\l\{\prod_{k=1}^{m}x_k:x_1,\cdots,x_m\in S\cup S^{-1},m\in\N_+\r\},\cdot\r).
    \end{align*}
\end{proof}
\begin{definition}[有限生成的,循环群]\label{xhq}
    设$G$是群.若$<S>=G$,则称$S$为$G$的一组生成元.若$G$中存在一有限集合$S$使得$<S>=G,$则称$G$为有限生成的.由一个元素生成的群称为循环群.
\end{definition}
\begin{proposition}
    有限群一定是有限生成的,反之未必成立.
\end{proposition}
\begin{proof}
    设$G$是有限群,$m=|G|\in\N_+$,则$G$所含元素的个数为$m$.取\begin{align*}
    S=\l\{g_1,g_2,\cdots,g_m:\text{$g_k$是$G$中两两不同的元素,$k=1,2,\cdots,m$}\r\},
    \end{align*}则\begin{align*}
        G=<S>,
    \end{align*}第一个论断证毕;

    注意到\begin{align*}
        &|(\Z,+)|=\infty\\
        &(\Z,+)=<\{1\}>,
    \end{align*}所以第二个论断证必.
\end{proof}
\begin{definition}[换位子,换位子群]
    对$\forall a,b\in G$,元素$a^{-1}b^{-1}ab$称为群$G$中元素$a,b$的换位子,简记为$[a,b]$.由所有换位子生成的群称为$G$的换位子群,记作$G^{(1)}$.
\end{definition}
\begin{proposition}
    已知$\varphi:G\to G'$是同态映射,有以下结论成立

    (1)若$G$是Abel群,则$\im\varphi$是Abel群;

    (2)虽$\im\varphi$是Abel群,但$G$未必是Abel群;

    (3)$\im\varphi$是Abel群$\iff G^{(1)}\subset\ker\varphi$.
\end{proposition}
\begin{proof}
    (1)

    对$\forall g_1,g_2\in G$\begin{align*}
    &\varphi(g_1)\varphi(g_2)\\
    \xlongequal[]{\varphi\text{是同态映射}}&\varphi(g_1g_2)\\
    \xlongequal[]{G\text{是Abel群}}&\varphi(g_2g_1)\\
    =&\varphi(g_2)\varphi(g_1),
    \end{align*}可见$\im\varphi$是Abel群.

    (2)

    令\begin{align*}
        &G=\l(\l\{\l(\begin{matrix}
        a&b\\
        c&d
        \end{matrix}\r):a,b,c,d\in\R\text{且}\l|\begin{matrix}
        a&b\\
        c&d
        \end{matrix}\r|\neq0\r\},\cdot\r)
    \end{align*}\begin{align*}
        \varphi:G\to&G'\\
        \l(\begin{matrix}
            a&b\\
            c&d
            \end{matrix}\r)\mapsto&\l(\begin{matrix}
            1&0\\
            0&1
        \end{matrix}\r),
    \end{align*}不难得到上述定义的$G$与$\varphi$证明了(2).

    (3)\begin{align*}
        &\im\varphi\text{是Abel群}\\
        \iff&\forall g_1,g_2\in G,\varphi(g_1)\varphi(g_2)=\varphi(g_2)\varphi(g_1)\\
        \iff&\forall g_1,g_2\in G,\l[\varphi(g_2)\varphi(g_1)\r]^{-1}\varphi(g_1)\varphi(g_2)=e\\
        \iff&\forall g_1,g_2\in G,\varphi(g_1^{-1}g_2^{-1}g_1g_2)=e\\
        \iff&\forall g_1,g_2\in G,g_1^{-1}g_2^{-1}g_1g_2\in\ker\varphi\\
        \iff&G^{(1)}\subset\ker\varphi,
    \end{align*}问题得证.
\end{proof}
\section{循环群}
循环群的定义见定义\ref{xhq}.
\begin{theorem}\label{p62dl4}
    整数加群$\Z$的子群都是由某一非负整数$m$生成的循环群.且$\forall m,n\in\N_+$\begin{align*}
        n\Z\supset m\Z\iff n|m.
    \end{align*}
\end{theorem}
\begin{proof}
    设$H$是$\l(\Z,+\r)$的一个子群.

    (1)(i)若$H=\l(\l\{0\r\},+\r)$,取$m=0$即可.

    (ii)若$H\neq\l(\l\{0\r\},+\r)$,则$H$中含有非零数,因而含有正整数.设$m$是$H$中最小的正整数,我们来证明$H=m\Z$.任取$x\in H$,由整数的除法算式有\begin{align*}
        x=qm+r,\text{其中$q,r\in\Z,0\leq r<m$},
    \end{align*}从而\begin{align*}
        r=x-qm\in H.
    \end{align*}若$r\neq0$,则$r$是$H$中小于$m$的正整数,矛盾,所以$r=0$,即\begin{align*}
        x=qm,
    \end{align*}这说明$H$中的任意元素都是$m$的倍数.反之,由群对运算的封闭性不难得到$m$的倍数也在$H$中.综上可得\begin{align*}
        H=m\Z,
    \end{align*}第一个论断证毕.

    (2)由$n\Z\supset m\Z$可知$m\in n\Z$,所以$n|m$;

    由$n|m$可知,$m$的倍数都是$n$的倍数,所以$n\Z\supset m\Z$.
    
    综上可得\begin{align*}
        n\Z\supset m\Z\iff n|m,
    \end{align*}第二个论断证毕.
\end{proof}
\begin{definition}[无限循环群]
    设群$G=<g>$,若$G$是无限群,则称$G$为无限循环群,此时记$|G|=\infty$.
\end{definition}
\begin{theorem}
    已知群$G=<g>,|G|=m$,则有以下结论成立

    (1)若$m=\infty,$则$G\cong\Z$,它的子群与非负整数成一一对应(见定理\ref{p62dl4});
    
    (2)若$m\in\N_+$,则$G\cong\Z/m\Z$,它的子群与$m$的因子一一对应.
\end{theorem}
\begin{proof}
    定义映射\begin{align*}
        \varphi:\Z&\to G\\
        n&\mapsto g^n,
    \end{align*}显然$\varphi$是良定义的满同态.
    
    (1)若$m=\infty$,则\begin{align*}
        &\forall n, m\in \N_+,n\neq m,\Lra g^n\neq g^m\\
        \iff&\varphi\text{是单射}\\
        \iff&\varphi\text{是同构映射}\\
        \iff&G\cong\Z.
    \end{align*}

    \stars
    
    (2)若$m\in\N_+$,不难得到\begin{align*}
        \ker\varphi=m\Z,
    \end{align*}于是根据群同态基本定理(定理\ref{qttjbdy})有\begin{align*}
        G=\im\varphi\cong\Z/k\Z.
    \end{align*}

    \stars
\end{proof}
\begin{lemma}\label{p63yl}
    设交换群$G$中元素$g,h$的阶为$m,n$且$(m,n)=1$,则元素$gh$的阶为$mn$.
\end{lemma}
\begin{proof}
    \stars
\end{proof}
\begin{theorem}\label{p64dy6}
    若$G$是有限交换群,则在$G$中存在一个元素,它的阶是$G$中所有元素阶的倍数.
\end{theorem}
\begin{proof}
    \stars
\end{proof}
\begin{theorem}\label{p64dl7}
    若$G$是有限交换群,则$G$是循环群的充分必要条件是对$\forall m\in\N_+$,在$G$中适合方程$x^m=e$的元素的个数不超过$m$.
\end{theorem}
\begin{proof}
    \stars
\end{proof}
\section{单群与\os{$A_n$}的单性}
\begin{definition}[单群]
    若群$G$没有非平凡的子群,则称群$G$为单群.
\end{definition}
\begin{theorem}
    设$G$为交换群,$G\neq\l\{e\r\}$,则$G$为单群的充分必要条件是$G$为素数阶的循环群.
\end{theorem}
\begin{proof}
    \stars
\end{proof}
\begin{definition}[置换]
    设$\varOmega$为有限集合,由$\varOmega$到自身的一个双射叫作$\varOmega$的一个置换.
\end{definition}
\begin{definition}[轮换]
    若一个$n$元置换$\sigma$把$i_1$映成$i_2$,把$i_2$映成$i_3$,$\cdots$,把$i_{r-1}$映成$i_r$,把$i_r$映成$i_1$,其余的元素保持不变,则称$\sigma$为一个$r$-轮换,简称轮换.$2$-轮换也称为对换.
\end{definition}
\begin{lemma}\label{p65yl}
    每个置换都可以表示成一些对换的乘积;每个偶置换(置换$\sigma$为偶置换当且仅当$\sigma$的对换分解式中对换的个数为偶数)都可以表示成一些长度为$3$的轮换(简称$3$-轮换)的乘积.
\end{lemma}
\begin{proof}
    \stars
\end{proof}
\begin{definition}[全变换群]
    非空集合$\varOmega$到自身的所有双射组成的集合,对于映射的乘法成群,称它为集合$\varOmega$的全变换群,记作$S_n$.
\end{definition}
\begin{definition}[$n$元交错群]
    $S_n$中所有偶置换组成的集合,对于映射的乘法成群,称它为$n$元交错群,记作$A_n$.
\end{definition}
\begin{theorem}
    交错群$A_n,n\geq5$是单群.
\end{theorem}
\begin{proof}
    \stars
\end{proof}
\section{可解群}\label{jsndaf}
对任意群$G$而言,它的换位子群$G^{(1)}$是$G$的正规子群,即\begin{align*}
    G^{(1)}\lhd G,
\end{align*}再做$G^{(1)}$的换位子群$\l(G^{(1)}\r)^{(1)}$,记作$G^{(2)}$,就有\begin{align*}
    G^{(2)}\lhd G^{(1)}\lhd G,
\end{align*}以此类推可得\begin{align*}
    \cdots\lhd G^{(k)}\lhd G^{(k-1)}\lhd\cdots\lhd G^{(2)}\lhd G^{(1)}\lhd G.
\end{align*}

若$G$是有限群,这样的群列只有以下两种可能\begin{align*}
    (1)&\exists k\in\N_+\st G^{(k)}=G^{(k+1)}=\cdots\neq\l\{e\r\}\\
    (2)&\exists k\in\N_+\st G^{(k)}=\l\{e\r\}.
\end{align*}
\begin{definition}[可解群]
    设$G$是群.若\begin{align*}
        \exists k\in\N_+\st G^{(k)}=\l\{e\r\},
    \end{align*}则称$G$为可解群.
\end{definition}
\begin{theorem}
    群$G$是可解的当且仅当存在一递降的子群列\begin{align*}
        G=G_0>G_1>\cdots>G_s=\l\{e\r\},
    \end{align*}其中每个$G_i$是前一个$G_{i-1}$的正规子群,且商群$G_{i-1}/G_i$交换($i=1,\cdots,s$).
\end{theorem}
\begin{proof}
    \stars
\end{proof}
由群的第二同构定理(定理\ref{qddetgdl})可知,当群$N$是群$G$的正规子群时,商群$G/N$的正规子群与$G$中包含$N$的正规子群是一一对应的.因此,商群$G/N$是单群的充要条件为正规子群$N$不包含在另一个非平凡的正规子群中,即不存在$G$的正规子群$N_1,N_1\neq G,N_1\neq N$,且\begin{align*}
    N<N_1\lhd G,
\end{align*}具有此性质的正规子群$N$称为极大的.
\begin{theorem}
    有限群$G$是可解的的充分必要条件为存在递降的子群列\begin{align*}
        G=H_0\rhd H_1\rhd\cdots\rhd H_t=\l\{e\r\},
    \end{align*}其中商群$H_{i-1}/H_i(i=1,\cdots,t)$都是素数阶的循环群.
\end{theorem}
\begin{proof}
    \stars
\end{proof}
\section{群的自同构群}
\begin{definition}[自同构与自同构群]
    一个群到它自身的同构映射称为自同构映射,简称为自同构.群的全部自同构在变换下的乘法下成群,称为自同构群.群$G$的自同构群记作$\Aut(G)$.
\end{definition}
设$G$为群,$a\in G$为固定元素.定义\begin{align}
    \sigma_a:&G\to G\nonumber\\
    &g\mapsto aga^{-1}\nonumber,
\end{align}不难验证$\sigma_a\in\Aut(G)$.
\begin{definition}[内自同构与内自同构群]
    称$\sigma_a$这种由$G$中元素引起的自同构为内自同构.$a\mapsto\sigma_a$给出了群$G$到$\Aut(G)$的同态,$G$的同态像就是$G$的全体内自同构,它们组成$\Aut(G)$的子群,记作$\In(G)$,称为$G$的内自同构群.
\end{definition}
\begin{definition}[中心]
    对任意群$G$,与$G$的全体元素可交换的元素组成的集合称为$G$的中心,记作$Z(G)$.
\end{definition}
\begin{theorem}
    定义映射\begin{align*}
        f:&G\to\In(G)\\
        &a\mapsto\sigma_a,
    \end{align*}则\begin{align*}
        &\ker f=Z(G)\\
        &G/Z(G)\cong\In(G).
    \end{align*}
\end{theorem}
\begin{proof}
    \stars
\end{proof}
\begin{theorem}
    对任意群$G$,有\begin{align*}
        \In(G)\lhd\Aut(G).
    \end{align*}
\end{theorem}
\begin{proof}
    \stars
\end{proof}
\begin{definition}[外自同构群]
    对任意群$G$,称自同构群对于内自同构群的商群\begin{align*}
        \Aut(G)/\In(G)
    \end{align*}为$G$的外自同构群.
\end{definition}
\begin{theorem}
    对任意群$G$,若$Z(G)=\l\{e\r\}$,则\begin{align*}
        G\cong\In(G).
    \end{align*}此时我们可以认为$G<\In(G)$.
\end{theorem}
\begin{proof}
    \stars
\end{proof}
\begin{definition}[完全群]
    一个中心为单位且自同构全是内自同构的群称为完全群.
\end{definition}
\section{群作用}
\begin{definition}[群$G$在集合$X$上的作用]\label{qgzjhxudzy}
    设$G$是群,$X$是非空集合.若映射$f:G\times X\to X$适合以下条件\begin{align*}
        \forall g_1,g_2\in G,&x\in X:\\
        &(1)f(e,x)=x\\
        &(2)f(g_1g_2,x)=f(g_1,f(g_2,x)),
    \end{align*}就称$f$决定了群$G$在集合$X$上的作用.
\end{definition}
\begin{remark}
    在不需要明确指出映射$f$情况下,通常把$f(g,x)$简写成$g(x)$.按此写法,定义\ref{qgzjhxudzy}中的条件就可以写成\begin{align*}
        (1)&e(x)=x\\
        (2)&g_1(g_2(x))=g_1g_2(x).
    \end{align*}
\end{remark}
\begin{example}\label{sahuidfj}
    设$G$是群,取$X=G$.定义\begin{align*}
        g(x)=gx,\ \ \text{对}g,x\in G.
    \end{align*}这就给出了一个群在集合$G$上的作用.此即以前所谓的左平移.
\end{example}
\begin{example}[共轭变换]\label{gebh}
    设$G$是群,取$X=G$.定义\begin{align*}
        g(x)=gxg^{-1},\ \ \text{对}g,x\in G.
    \end{align*}此即群$G$上的共轭变换.称元素$x$与元素$gxg^{-1}$共轭;称子群$H$与子群$gHg^{-1}$共轭,它们都是等价关系;称群$G$共轭作用在集合$G$上.
\end{example}
\begin{example}\label{sdghbj}
    设$G$是群,$H<G$,令$X=\l\{xH:x\in G\r\}$,定义\begin{align*}
        g(xH)=gxH,\ \ g,x\in G.
    \end{align*}这就决定了群$G$在集合$X$上的左右.
\end{example}
\begin{definition}[齐性空间]
    设$G$是群,$H<G$,称\begin{align*}
        X=\l\{xH:x\in G\r\}
    \end{align*}是群$G$的一个齐性空间.
\end{definition}
当群$G$作用在集合$X$上时,有可能$G$中的不同的元素在$X$上引起相同的映射,亦即$g\mapsto\sigma_g$不一定是单射.比如在例\ref{gebh}中,$Z(G)$中的元素都对应$G$上的恒同映射.
\begin{definition}[如实的]
    若映射\begin{align*}
        f:&G\to\In(G)\\
        &a\mapsto\sigma_a
    \end{align*}是单射,就称群$G$在集合$X$上的作用是如实的,或称群$G$如实地作用在集合$X$上.其中\begin{align}
        \sigma_a:&G\to G\nonumber\\
        &g\mapsto aga^{-1}\nonumber.
    \end{align}
\end{definition}
\begin{remark}
    例\ref{sahuidfj}中的作用是如实的;例\ref{gebh},例\ref{sdghbj}不一定是如实的.
\end{remark}
\begin{definition}[等价的]
    设$G$是群,$X$与$X'$是非空集合,$G$作用在$X$与$X'$上.若有一个一一对应$\varphi:X\to X'$使得\begin{align*}
        \varphi\l(g(x)\r)=g\l(\varphi(x)\r),
    \end{align*}则称$G$在集合$X$与$X'$上的作用是等价的.
\end{definition}
从抽象的观点来看,两个等价的作用可以不加区别.
\begin{definition}[集合$X$上的等价关系]\label{jhxuddjgx}
    对$\forall x,y\in X$,若\begin{align*}
        \exists g\in G\st y=g(x),
    \end{align*}则称$x$等价于$y$,记作$x\sim y$.
\end{definition}
\begin{remark}
    定义\ref{jhxuddjgx}中的关系$\sim$是等价关系的证明如下.
    \begin{proof}
        \stars
    \end{proof}
\end{remark}
\begin{definition}[$G$-轨道]
    在定义\ref{jhxuddjgx}中的等价关系下,集合$X$中的元素被分成等价类,称这样分成的等价类为$x$的$G$-轨道,简称轨道,记作$O_x$.称$x$为该轨道的代表.
\end{definition}
\begin{remark}\label{sdaujnf}
    由于轨道就是等价类,所以任意两条轨道要么相等,要么无交,即\begin{align*}
        X=\bigcup_{i\in I}O_{x_i},
    \end{align*}其中当$x_i\neq x_j$时$O_{x_i}\cap O_{x_j}=\varnothing$.进而\begin{align*}
        |X|=\sum_{i\in I}|O_{x_i}|.
    \end{align*}
\end{remark}
\begin{definition}[完全代表系]
    称集合\begin{align*}
        \l\{x_i:i\in I\r\}
    \end{align*}为$x$的$G$-轨道的完全代表系.
\end{definition}
\begin{definition}[不动元素]
    当轨道$O_x$只含有一个元素$x$即\begin{align*}
        \forall g\in G,g(x)=x
    \end{align*}时,称$x$为$G$的不动元素.
\end{definition}
在例\ref{gebh}中,若$x\in Z(G)$,则显然$O_x=\l\{x\r\}$;反之,由$O_x=\l\{x\r\}$可知$x\in Z(G)$.
\begin{definition}[传递的]
    设群$G$作用在集合$X$上,当$X$是一个轨道即\begin{align*}
        \forall x,y\in X,\exists g\in G\st g(x)=g(y)
    \end{align*}时,称群$G$在集合$X$上的作用是传递的.
\end{definition}
不难发现,例\ref{sahuidfj}与例\ref{sdghbj}都是传递的的情形.
\begin{definition}[稳定子与稳定子群]
    设群$G$作用在集合$X$上,对$\forall x\in X$,称集合\begin{align*}
        H_x=\l\{g\in G:g(x)=x\r\}
    \end{align*}是$x$的稳定子.容易验证,$H_x$是$G$的子群,因此也称为元素$x$的稳定子群.
\end{definition}
当群$G$在集合$G$上的作用是共轭作用(参考例\ref{gebh})时\begin{align}
    H_x=&\l\{g\in G:gxg^{-1}=x\r\}\nonumber\\
    =&\l\{g\in G:gx=xg\r\}.\label{sdaubhf}
\end{align}
\begin{definition}[中心化子]\label{vxhz}
    称等式\eqref{sdaubhf}右端的集合为$x$在$G$里的中心化子,记作$Z(x)$,它就是在群$G$的共轭作用下$x$的稳定子群$H_x$.
\end{definition}
\begin{theorem}[轨道-稳定子定理]\label{gdwdzdl}
    设群$G$作用在集合$X$上,$x\in X,O_x$是包含$x$的轨道,$H_x$是$x$的稳定子群,则群$G$在集合$O_x$上的作用与群$G$在齐性空间$G/H_x$上的作用等价,也可写作\begin{align*}
        |O_x|=[G:H_x],
    \end{align*}即$x$的轨道的长度($x$的轨道所含元素的个数)等于$x$的稳定子在$G$中的指数.
\end{theorem}
\begin{proof}
    \stars
\end{proof}
\begin{corollary}
    设群$G$在集合$X$上的作用是传递的,$x\in X,H_x$是元素$x$的稳定子群.则$G$在$X$上的作用与$G$在齐性空间$G/H_x$上的作用等价.
\end{corollary}
\begin{proof}
    \stars
\end{proof}
\begin{corollary}
    设有限群$G$作用在集合$X$上.则任意一个轨道$O_x$包含有限多个元素,并且包含的元素的个数是$|G|$的因子.
\end{corollary}
\begin{proof}
    \stars
\end{proof}
\begin{definition}[$p$-群]
    设$G$是有限群,若$|G|$是素数$p$的方幂,即\begin{align*}
        |G|=p^k,\ \ k\geq1,
    \end{align*}则称$G$为$p$-群($p$是素数).
\end{definition}
\begin{corollary}\label{p77tl3}
    设有限群$G$作用在有限集合$X$上.若$G$是$p$-群,$|X|=n,(n,p)=1$,则$X$中一定有不动元素.
\end{corollary}
\begin{proof}
    \stars
\end{proof}
\begin{corollary}\label{p77tl4}
    设$p$-群作用在有限集合$X$上,$|X|=n$.若$t$为$X$中不动元素的个数,则\begin{align*}
        t\equiv n\mod p.
    \end{align*}
\end{corollary}
\begin{proof}
    \stars
\end{proof}
\begin{corollary}
    $p$-群有非平凡的中心.
\end{corollary}
\begin{proof}
    \stars
\end{proof}
\begin{definition}[共轭类]
    当群$G$在集合$G$上的作用是共轭变换(例\ref{gebh})时,称轨道$O_x$为$x$所在的共轭类,记作$C(x)$.
\end{definition}
\begin{theorem}
    设群$G$作用在集合$X$上,$x,y\in X$.若存在$g_0\in G$使得$y=g_0x$,则$H_y=g_0H_xg_0^{-1}$.
\end{theorem}
\begin{proof}
    \stars
\end{proof}
\section{Sylow定理}
Lagrange定理(推论\ref{Lagrangedl})指出,有限群$G$的任意子群的阶是$|G|$的因子.反之,对于$|G|$的任意正因子$d$,是否存在一个$d$阶子群?本节介绍的Sylow定理将回答这一问题.

在此之前我们需要以下引理.
\begin{lemma}\label{p79yl}
    若$n=p^lm,(p,m)=1,k\leq l,C_n^{p^k}$是组合数,则\begin{align*}
        p^{l-k}|C_n^{p^k},p^{l-k+1}\nmid C_n^{p^k}.
    \end{align*}
\end{lemma}
\begin{proof}
    \stars
\end{proof}
\begin{theorem}[Sylow第一定理]\label{xldydl}
    若群$G$的阶为$n=p^lm$,其中$p$为素数,$(p,m)=1,l\geq1$,则对于$0\leq k\leq l$,$G$有$p^k$阶子群.特别地,称$p^l$阶子群为$G$的$\mr{Sylow}\ p$-子群.
\end{theorem}
\begin{proof}
    令\begin{align*}
        X=\l\{A:A\subset G,|A|=p^k\r\},
    \end{align*}易见$|X|=C_{n}^{p^k}$.对$\forall A\in X$,对$\forall g\in G$定义映射\begin{align*}
        g:&X\to X\\
        &A\mapsto gA,
    \end{align*}不难验证这是良定义的.该映射给出了群$G$在集合$X$上的作用.

    由注\ref{sdaujnf}可知\begin{align*}
        \l|X\r|=\sum_{i\in I}\l|O_{A_i}\r|.
    \end{align*}由引理\ref{p79yl}可知\begin{align*}
        p^{l-k+1}\nmid C_n^{p^k}=|X|,
    \end{align*}所以至少有一个轨道,不妨设为$O_{A_j}$,满足\begin{align*}
        p^{l-k+1}\nmid|O_{A_j}|,
    \end{align*}此即$|O_{A_j}|$含有的$p$因子至多为$p^{l-k}$.
    
    对于$A_j$的稳定子群$H_{A_j}$,由轨道-稳定子定理(定理\ref{gdwdzdl})可知\begin{align*}
        |O_{A_j}|=[G:H_{A_j}]=\frac{|G|}{|H_{A_j}|},
    \end{align*}注意到\begin{align*}
        &\text{$|G|$含有的$p$因子恰好为$p^l$}\\
        &\text{$|O_{A_j}|$含有的$p$因子至多为$p^{l-k}$},
    \end{align*}所以\begin{align*}
        \text{$H_{A_j}$含有的$p$因子至少为$p^k$},
    \end{align*}亦即\begin{align*}
        \exists q\in\N_+\st|H_{A_j}|=p^kq,
    \end{align*}可见\begin{align}\label{duhsbvj}
        |H_{A_j}|\geq p^k.
    \end{align}

    反之,对$\forall g\in H_{A_j}$,有$g(A_j)=A_j$,所以对$\forall a\in A_j$,有$ga\in A_j$,从而\begin{align*}
        H_{A_j}a=\l\{ga:g\in H_{A_j}\r\}\subset A_j,
    \end{align*}于是\begin{align}\label{sduibfhbj}
        |H_{A_j}|=\l|H_{A_j}a\r|\leq|A_j|=p^k.
    \end{align}

    由式\eqref{duhsbvj}与式\eqref{sduibfhbj}可知,$H_{A_j}$是$G$的$p^k$阶子群.定理得证.
\end{proof}
\begin{theorem}[Sylow第二定理]\label{xldedl}
    若有限群$G$的阶为$p^lm$,其中$p$为素数且$(p,m)=1$,记$P$为$G$的$\mr{Sylow}\ p$-子群.则$G$的任意一个阶为$p^k(k\leq l)$的子群$H$包含在一个与$P$共轭的$\mr{Sylow}\ p$-子群中.
\end{theorem}
\begin{proof}
    令\begin{align*}
        X=\l\{gP:g\in G\r\},
    \end{align*}对$\forall h\in H$,定义映射\begin{align*}
        h:X\to&X\\
        gP\mapsto&hgP,
    \end{align*}容易验证该定义是良定义的,$h$确定了群$H$在集合$X$上的作用.由Lagrange定理(推论\ref{Lagrangedl})的证明过程可知\begin{align*}
        |G|=|X|\cdot|P|,
    \end{align*}所以$|X|=m$.注意到有限$p$-群$H$作用在集合$X$上,并且$(p,m)=1$,所以由推论\ref{p77tl3}可知$X$有不动元素,不妨设其中的一个为$g_jP$,即\begin{align*}
        &\forall h\in H,hg_jP=g_jP\\
        \iff&\forall h\in H,g_j^{-1}hg_jP=P\\
        \iff&\forall h\in H,g_j^{-1}hg_j\in P\\
        \iff&\forall h\in H,h\in g_jPg_j^{-1}\\
        \iff&H\subset g_jPg_j^{-1},
    \end{align*}注意到$H,P$均成群,所以$H<g_jPg_j^{-1}$,即$H$包含在一个与$P$共轭的\Sl{p}中,定理得证.
\end{proof}
\begin{corollary}\label{p81tl1}
    对有限群而言,任意两个\Sl{p}都互相共轭.
\end{corollary}
\begin{proof}
    \stars
\end{proof}
\begin{corollary}\label{asdinfjkasdfmk}
    有限群$G$的\Sl{p}是惟一的当且仅当$G$的\Sl{p}是正规子群.
\end{corollary}
\begin{proof}
    $G$的子群$P$是正规子群即$P$的所有共轭子群都等与$P$自身.

    任取$G$的两个\Sl{p}$P_0,P$,由推论\ref{p81tl1}可知$P_0,P$互相共轭,又由于$P_0,P$互相共轭可知$P_0=P$,推论得证.
\end{proof}
\begin{definition}[正规化子]\label{vghz}
    对群$G$的任意子群$H$,定义\begin{align*}
        N_G(H)=\l\{g\in G:gHg^{-1}=H\r\},
    \end{align*}则$N_G(H)<G$且$H\subset N_G(H)$.称$N_G(H)$为子群$H$在$G$中的正规化子,简记为$N(H)$.
\end{definition}
\begin{remark}
    由定义\ref{vghz}可立即推出$H\lhd N_G(H)$,这也是引出正规化子的意义.
\end{remark}
\begin{corollary}\label{p81tl2}
    若$G$是有限群,$P$是$G$的\Sl{p},则\begin{align*}
        (1)&N_G(N_G(P))=N_G(P)\\
        (2)&\text{$N_G(P)$不包含$G$的另一个\Sl{p}.}
    \end{align*}
\end{corollary}
\begin{proof}
    \stars
\end{proof}
\begin{corollary}\label{p81tl3}
    若$G$是有限群,$p$是素数且$p\Big||G|$,则$G$的\Sl{p}的个数是$|G|$的因子.
\end{corollary}
\begin{proof}
    \stars
\end{proof}
\begin{theorem}[Sylow第三定理]\label{xldsdl}
    若$|G|=p^lm,p$是素数且$(p,m)=1,l\geq1$,记$G$的\Sl{p}的个数为$k$,则$k\equiv1\mod p$.
\end{theorem}
\begin{analysis}
    注意到该定理的结论与推论\ref{p77tl4}十分相似,这引诱我们构造某$p$-群$P$,使其作用在某$k$元集合$X$后所得不动元素的个数为$1$.
\end{analysis}
\begin{proof}
    令\begin{align*}
        X=\l\{P:P<G,|P|=p^l\r\},
    \end{align*}于是$|X|=k$.

    任取$P\in X$,考虑$P$在$X$上的共轭作用.对$\forall a\in P$,定义映射\begin{align*}
        a:X\to&X\\
        Q\mapsto&aQa^{-1},
    \end{align*}容易验证这是良定义的,$a$给出了群$P$在集合$X$上的作用.设$X_0$为$P$作用在$X$上的不动点集,即\begin{align*}
        X_0=\l\{Q\in X:a(Q)=Q\r\},
    \end{align*}则\begin{align*}
        Q\in X_0\iff&\forall a\in P,a(Q)=Q\\
        \iff&\forall a\in P,aQa^{-1}=Q\\
        \iff&\forall a\in P,a\in N(Q)\\
        \iff&P\subset N(Q),
    \end{align*}又$Q\subset N(Q)$且$P,Q$都是$G$的\Sl{p},所以由推论\ref{p81tl2}可知$P=Q$,进而$|X_0|=1$.由推论\ref{p77tl4}可知\begin{align*}
        |X_0|\equiv|X|\mod p,
    \end{align*}即\begin{align*}
        &1\equiv k\mod p\\
        \iff&k\equiv1\mod p,
    \end{align*}定理得证.
\end{proof}
\begin{corollary}\label{p82tl}
    若群$G$的阶为$p^lm$,其中$p$是素数且$(p,m)=1$,则$G$的\Sl{p}的个数是$m$的因子.
\end{corollary}
\begin{proof}
    设$G$的\Sl{p}的个数为$k$.

    由推论\ref{p81tl3}可知$k\Big||G|$,由$|G|=p^lm$,$p$是素数且$(p,m)=1$\begin{align}\label{asdkmfjn}
        k|p\text{或}k|m.
    \end{align}

    由Sylow第三定理(定理\ref{xldsdl})可知\begin{align}\label{sadijf}
        k\nmid p.
    \end{align}

    由式\eqref{asdkmfjn}与\eqref{sadijf}可知$k|m$,推论得证.
\end{proof}
作为本节的结束,我们来看一个例子来说明如何利用上面的结果来解决群论的问题.
\begin{example}\label{nvkkadv}
    已知有限群$G$的阶为$72$,证明$G$不是单群.
\end{example}
\begin{proof}
    注意到\begin{align*}
        72=2^3\cdot3^2,
    \end{align*}所以利用Sylow第三定理(定理\ref{xldsdl}),可设$G$的Sylow\ $3$-子群的个数为\begin{align*}
        1+3t,t\in\N,
    \end{align*}再利用推论\ref{p82tl}可知\begin{align*}
        (1+3t)|2^3,
    \end{align*}所以$t=0$或$t=1$.

    (1)若$t=0$,即$G$有惟一的Sylow\ $3$-子群,设为$P$,由推论\ref{asdinfjkasdfmk}可知\begin{align*}
        P\lhd G,
    \end{align*}所以此时$G$有非平凡的正规子群$P$,从而不是单群.

    (2)若$t=1$,即$G$有$4$个Sylow\ $3$-子群,设为$P_1,P_2,P_3,P_4$.考虑$G$在集合\begin{align*}
        X=\l\{P_1,P_2,P_3,P_4\r\}
    \end{align*}上的共轭变换(例\ref{gebh}).由推论\ref{p81tl1}可知$G$任意两个Sylow\ $3$-子群都互相共轭,所以$G$的每个元素都在$X$上诱导出一个$4$次置换,从而诱导出同态\begin{align*}
        \phi:G\to S_4,
    \end{align*}由群同态基本定理(定理\ref{qttjbdy})可知\begin{align*}
        G/\ker\phi\cong\im\phi,
    \end{align*}所以\begin{align}
        &|\ker\phi|=\frac{|G|}{|\im\phi|}\geq\frac{|G|}{|S_4|}=3\nonumber\\
        \Lra&\ker\phi\neq\l\{e\r\}.\label{sadfnj}
    \end{align}由$G$的Sylow\ $3$-子群不惟一可知\begin{align}
        &\im\phi>1\nonumber\\
        \Lra&\ker\phi\neq\l\{e\r\}\nonumber\\
        \Lra&\ker\phi\neq G.\label{sdjfnkajs}
    \end{align}由式\eqref{sadfnj}与\eqref{sdjfnkajs}可知,此时$\ker\phi$是$G$的非平凡的正规子群.

    综上,$G$不是单群.
\end{proof}
\begin{remark}
    在证明例\ref{nvkkadv}时所构造的群同态$\phi$是十分重要的,通常将其称为传递置换表示,下面严格地给出它的定义.
\end{remark}
\begin{definition}[传递置换表示]\label{idvhbu}
    若$H$是群$G$的子群,$H$的全体左陪集构成集合\begin{align*}
        P=\l\{a_iH:a_i\in G,i=1,\cdots,r\r\},
    \end{align*}则称群作用\begin{align*}
        g:P\to&P\\
        a_iH\mapsto&ga_iH
    \end{align*}所诱导出的群同态\begin{align*}
        \phi:G\to S_r
    \end{align*}为群$G$在子群$H$上的传递置换表示.
\end{definition}
\section{群的直和}
现在来介绍一种由已知群来构造新群的方法.先看两个群的情形.

设$G_1,G_2$成群,考虑集合\begin{align*}
    G_1\times G_2,
\end{align*}对$\forall G_1\times G_2$中的两个元素$(a_1,b_1),(a_2,b_2)$,定义乘法为\begin{align}
    (a_1,b_1)(a_2,b_2)=(a_1a_2,b_1b_2),\label{asjdkgnfasdm}
\end{align}其中第一个分量为作$G_1$的乘法,第二个分量为作$G_2$的乘法.

若$e_1,e_2$分别为$G_1,G_2$中的幺元,则可验证$G_1\times G_2$在新定义的乘法下成群,幺元为$(e_1,e_2)$.
\begin{definition}[群的直和]\label{qdzh}
    已知$G_1,G_2$成群,则将集合$G_1\times G_2$在乘法\eqref{asjdkgnfasdm}下所成的群称为$G_1$与$G_2$的直和,记作\begin{align*}
        G_1\oplus G_2.
    \end{align*}
\end{definition}
容易验证有以下$3$个命题成立.
\begin{proposition}\label{qdzhmt1}
    当群$G_1,G_2$是有限群时,$G_1\oplus G_2$也是有限群,并且\begin{align*}
        |G_1\times G_2|=|G_1|\cdot|G_2|.
    \end{align*}\qed
\end{proposition}
\begin{proposition}\label{qdzhmt2}
    在$G_1\oplus G_2$中令\begin{align*}
        &\overline{G_1}=\l\{a,e_2:a\in G_1\r\},\ \ \text{其中$e_2$为$G_2$中的幺元}\\
        &\overline{G_2}=\l\{e_1,b:b\in G_2\r\},\ \ \text{其中$e_1$为$G_1$中的幺元},
    \end{align*}则\begin{align*}
        &\ol{G_1}\lhd G_1\times G_2,\ol{G_2}\lhd G_1\times G_2\\
        &\ol{G_1}\cong G_1,\ol{G_2}\cong G_2.
    \end{align*}\qed
\end{proposition}
\begin{proposition}\label{qdzhmt3}
    $G_1\oplus G_2$中的每个元素都可以分解成$\ol{G_1}$与$\ol{G_2}$中的元素的乘积,并且该分解是唯一的.\qed
\end{proposition}
定义\ref{qdzh}与命题\ref{qdzhmt1},\ref{qdzhmt2},\ref{qdzhmt3}均可以推广到多个群的情形.

经过上述讨论我们已经知道,直和$\bigoplus_{i=1}^{s}G_i$的结构完全被群$G_i$的结构决定.因此如果一个群能够分解成一些群的直和,那么该群的研究就可以归结为另一些群(一般比原来的群简单)的研究.下面将讨论在什么情况下,一个群能够分解成一些群的直和.
\begin{theorem}\label{p85dl19}
    设群$N_i$是群$G$的正规子群,$i=1,\cdots,s$,若\begin{align*}
        &(1)\ G=\prod_{i=1}^{s}N_i\\
        &(2)\ \text{对$\forall g\in G$,表示式$g=\prod_{i=1}^{s}g_i$,是唯一的,其中$g_i\in N_i$},
    \end{align*}则\begin{align*}
        G\cong\bigoplus_{i=1}^{s}N_i.
    \end{align*}
\end{theorem}
\begin{proof}
    由表示式$g=\prod_{i=1}^{s}$的唯一性可知映射\begin{align*}
        \phi:G\to&\prod_{i=1}^{s}\\
        g_1g_2\cdots g_s\mapsto&\l(g_1,g_2,\cdots,g_s\r)
    \end{align*}是一一对应.若能证明$\phi$是同态映射,便能得到$\phi$是同构映射,进而$G\cong\prod_{i=1}^{s}N_s$.
    
    注意到\begin{align}
        &\text{$\phi$是同态映射}\nonumber\\
        \iff&\forall x,y\in G,\phi(xy)=\phi(x)\phi(y)\nonumber\\
        \iff&\forall\prod_{i=1}^{s}x_i,\prod_{i=1}^{s}y_i\in G,\phi\l[\l(\prod_{i=1}^{s}x_i\r)\cdot\l(\prod_{i=1}^{s}y_i\r)\r]=\phi\l(\prod_{i=1}^{s}x_i\r)\cdot\phi\l(\prod_{i=1}^{s}y_i\r)\nonumber\\
        \iff&\forall\prod_{i=1}^{s}x_i,\prod_{i=1}^{s}y_i\in G,\phi\l[(x_1x_2)\cdots(x_sy_1)\cdots(y_{s-1}y_s)\r]=\nonumber\\
        &\phi\l(\prod_{i=1}^{s}x_i\r)\cdot\phi\l(\prod_{i=1}^{s}y_i\r)\nonumber\\
        \iff&\forall\prod_{i=1}^{s}x_i,\prod_{i=1}^{s}y_i\in G,(x_1x_2,\cdots,x_sy_1,\cdots x_sy_s)=(x_1,\cdots x_s)(y_1,\cdots y_s)\nonumber\\
        \iff&\forall\prod_{i=1}^{s}x_i,\prod_{i=1}^{s}y_i\in G,(x_1x_2,\cdots,x_sy_1,\cdots x_sy_s)=(x_1y_1,\cdots,x_sy_s)\nonumber\\
        \iff&\forall\prod_{i=1}^{s}x_i,\prod_{i=1}^{s}y_i\in G,x_1x_2\cdots x_sy_1\cdots y_{s-1}y_s=x_1y_1\cdots x_sy_s\nonumber\\
        \iff&\forall g_i,g_j\in G,g_ig_j=g_jg_i,\label{adisjofhubj}
    \end{align}可见,问题转化成了证明\eqref{adisjofhubj}.

    对$\forall g\in N_i\cap N_j,i\neq j$,假设$g\neq e$,则$g$有两种不同的表示方式\begin{align*}
        \begin{tikzpicture}
        \node at (0, 0) {$g = e \cdots e g e \cdots e = e \cdots e g e \cdots e$};
        \draw [<-] (-0.9, -0.1) -- (-0.9, -1) node[below] {第 $i$ 位};
        \draw [<-] (1.6, -0.1) -- (1.6, -1) node[below] {第 $j$ 位};
        \end{tikzpicture}
    \end{align*}亦即$G$中的元素$g$有两种不同的分解方式,这与题设条件(2)矛盾,所以假设不成立,即\begin{align*}
        N_i\cap N_j=\l\{e\r\},i\neq j.
    \end{align*}
    
    对$\forall g_i\in N_i,g_j\in N_j\lhd G,i\neq j$\begin{align*}
        N_i\text{成群}\Lra&g_i^{-1}\in N_i\\
        g_i^{-1}\in N_i\lhd G,g_j\in N_j\lhd G\xLra[]{\text{定义}\ref{sadjfgnka}}&g_jg_i^{-1}g_j^{-1}\in N_i,
    \end{align*}从而\begin{align}
    g_ig_jg_i^{-1}g_j^{-1}\in N_i,\label{asdnifjx}
    \end{align}同理可证\begin{align}
        g_ig_jg_i^{-1}g_j^{-1}\in N_j.\label{sajdnjbf}
    \end{align}由\eqref{asdnifjx}与\eqref{sajdnjbf}可知\begin{align}
        &g_ig_jg_i^{-1}g_j^{-1}\in N_i\cap N_j=\l\{e\r\}\nonumber\\
        \iff&g_ig_jg_i^{-1}g_j^{-1}=e.\label{asdjfk}
    \end{align}注意到当$i=j$时\eqref{asdjfk}也成立,从而$\forall g_i,g_j\in G$\begin{align*}
        &g_ig_jg_i^{-1}g_j^{-1}=e\\
        \iff&g_ig_j=g_jg_i,
    \end{align*}可见\eqref{adisjofhubj}成立,从而定理得证.
\end{proof}
\begin{definition}[内直和]\label{nzh}
    若群$G$同构于其正规子群$N_1,\cdots,N_s$的直和,则称群$G$分解成正规子群$N_1,\cdots,N_s$的直和,也称$G$等于$N_1,\cdots,N_s$的内直和.
\end{definition}
\begin{proposition}\label{asdjn}
    与线性空间分解成子空间的直和的情况类似,不难证明定理\ref{p85dl19}的条件$(2)$有以下两种等价描述
    
    (2')幺元的表示方式唯一,即\begin{align*}
        \text{若$e=\prod_{i=1}^sx_i$,则$x_i=e$}
    \end{align*}
    
    (2'')\begin{align*}
        N_j\bigcap\prod_{i=1,i\neq j}^{s}N_i=\l\{e\r\}.
    \end{align*}
\end{proposition}
\begin{definition}[不可分解的]
    若群$G$不能被分解成两个非平凡的正规子群的直和,则称$G$是不可分解的.
\end{definition}
事实上,任意一个有限群总能分解成一些不可分解的群的直和.群的直和是群论中的重要问题,这里不再细说.

下面给出一个例子,来看看有限交换群的分解.
\begin{example}
    有限交换群能被分解成$p$-群的直和.
\end{example}
\begin{proof}
    设$G$是有限交换群,$|G|=n$的标准分解式为\begin{align*}
        n=\prod_{i=1}^{r}p_i^{r_s},
    \end{align*}其中$p_i$是不同的素数,$r_i>0$.

    由Sylow第一定理(定理\ref{xldydl})可知$G$存在Sylow\ $p_i$-子群,记为$G_i$.注意到$G$是交换群,所以$G_i$是正规的,从而由推论\ref{asdinfjkasdfmk}可知$G_i$是$G$的惟一一个Sylow\ $p_i$子群,结合\ref{xldydl}不难看出$G_i$恰由$G$中所有阶为$p_i$的幂的元素组成.

    令$H=\prod_{i=1}^{s}G_i$,由
    \begin{align*}
        &\text{$G_i$中的元素的阶为$p_i$的方幂}\\
        &\text{子群$\prod_{i=1,i\neq j}^{s}G_i$元素的阶为$\prod_{i=1,i\neq j}^{s}p_i^{r_i}$的因子(引理\ref{p63yl})},
    \end{align*}可知\begin{align*}
        G_j\cap\prod_{i=1,i\neq j}^{s}G_i=\l\{e\r\},
    \end{align*}结合定理\ref{p85dl19}与命题\ref{asdjn}可得\begin{align*}
        H\cong\bigoplus_{i=1}^{s}G_i.
    \end{align*}

    由上述讨论可知\begin{align*}
        &H<G\\
        &|H|=|G|,
    \end{align*}所以$H=G$,进而\begin{align*}
        G\cong\bigoplus_{i=1}^{s}G_i,
    \end{align*}问题得证.
\end{proof}
\begin{remark}
    有时交换$p$-群还能被分解.以后将证明$p$-群不能被分解的充要条件为它是循环群.
\end{remark}
% \section{{\JH}定理}
% 本节将指出,递降子群列也是刻画一般有限群结构的重要工具.
% \begin{definition}[次正规子群列,因子群组,合成群列,长度]\label{czgzql}\label{yzqz}\label{hcql}
%     若群$G$的递降子群列\begin{align}
%         G=G_0>G_1>\cdots>G_r=\l\{e\r\}\label{njksdc}
%     \end{align}满足$G_{i-1}\rhd G_{i}$,则称\eqref{njksdc}为$G$的次正规子群列.

%     称\eqref{njksdc}的商群组\begin{align}
%         G_0/G_1,G_1/G_2,\cdots,G_{r-1}/G_r\label{saidufhj}
%     \end{align}为\eqref{njksdc}的因子群组.

%     若\eqref{njksdc}的因子群组\eqref{saidufhj}都是单群,则称\eqref{njksdc}为合成群列.

%     在\eqref{njksdc}中可能有重复项出现,即在\eqref{saidufhj}中可能有单位群$\l\{e\r\}$出现.称\eqref{saidufhj}中非单位群的因子群的个数为\eqref{njksdc}的长度.
% \end{definition}
% \begin{remark}
%     由于$G_{i-1}\rhd G_{i}\nRightarrow G_{i-1}\rhd G$,所以叫'次'正规.

%     群的次正规子群列不惟一,群的递降子群列的长度惟一.
% \end{remark}
% \begin{proposition}
%     有限群有合成群列.
% \end{proposition}
% \begin{proof}
%     \stars
% \end{proof}
% 在上述证明过程中可知,只要某一因子群$G_i/G_{i+1}$不是单群,则可在$G_i$与$G_{i+1}$之间插入一个子群$H$使次正规子群列的长度增加.亦即合成群列是不能再插入任何一项的次正规子群列.

% 无限群而言不一定有合成群列,如整数加法群.为了确保无限群存在合成群列,必须添加适当的条件,本书不作讨论.

% 在\ref{jsndaf}节中已经得到以下两条结论

% 1)每个有限可解群$G$有一个合成群列\begin{align*}
%     G=G_0\rhd G_1\cdots\rhd G_r=\l\{e\r\},
% \end{align*}使得每个因子群$G_i/G_{i+1}$都是素数阶循环群.反之也成立.

% 2)有限可解群$G$的任一合成群列(无重复项)$G=G_0\rhd G_1\rhd G_2\rhd\cdots\rhd G_r=\l\{e\r\}$的因子群组\begin{align*}
%     G_0/G_1,G_1/G_2,\cdots,G_r{-1}/G_r
% \end{align*}不及次序地由$G$惟一决定,与合成群列的选取无关.

% 第一条是有限可解群的特性,用它可以判别一个有限群是不是可解的.第二条则不是有限群的特性,它是一般有限群类的共性,这表现在下述定理中
% \begin{theorem}[{\JH}定理]\label{jhdl}
%     有限群的无重复项的合成群列有相同的长度,并且它们的因子群组在同构意义下不计次序地一一相等.
% \end{theorem}
% \begin{proof}
%     \stars
% \end{proof}
% {\JH}定理(定理\ref{jhdl})指出,给定的有限群的全体合成群列的因子群组是相同的,由一组非平凡的单群构成,因此也称其为该有限群的因子群组.反之,任给一组有限单群\begin{align*}
%     S=\l\{S_1,\cdots,S_r\r\},
% \end{align*}以$S$为因子群组的有限群有多少种不同构的类型?一般来说,它可以归结为以下问题
% \begin{*problem}
%     对任意群$H,N$,试构造出所有不同构的群$G_i$使得\begin{align*}
%         &N\lhd G_i\\
%         &G_i/H\cong H.
%     \end{align*}
% \end{*problem}
% \noindent 这就是所谓的群扩张问题.群扩张是群论中的重要理论,这里不作讨论.
\section{幺半群}
\stars