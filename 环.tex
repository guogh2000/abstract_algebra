第一章介绍了环,理想,商环以及环同态等概念并建立了环同态基本定理.本章将进一步给出环的几个重要的同态定理与几种构造环的方法.
\section{环的同态定理}
在给出环的同态定理之前,先介绍理想的运算.
\begin{proposition}\label{zqdj}
    子群的交仍是子群.
\end{proposition}
\begin{proof}
    设$H,N$是群$(G,\cdot)$的子群,现在来证明$H\cap N$仍是$G$的子群.

    $\forall x,y\in H\cap N$,易见\begin{align*}
        &x,y\in H\xLra[]{H<G}xy^{-1}\in H\\
        &x,y\in N\xLra[]{N<G}xy^{-1}\in N,
    \end{align*}从而\begin{align*}
        x,y^{-1}\in H\cap N\Lra H\cap N<G,
    \end{align*}命题得证.
\end{proof}
\begin{proposition}\label{jkasdnf}
    已知$H,N$是环$R$的子环,则$H\cap N$是环$R$的子环.
\end{proposition}
\begin{proof}
    由命题\ref{zqdj}可知$H\cap N$是$R$的加法子Abel群,所以只需证明$H\cap N$是关于$R$的乘法封闭即可.
    
    $\forall x,y\in H\cap N$注意到\begin{align*}
        x,y\in H\Lra&xy\in H\\
        x,y\in N\Lra&xy\in N,
    \end{align*}可见\begin{align*}
        \forall x,y\in H\cap N\Lra&xy\in H\cap N\\
        \iff&\text{$H\cap N$关于$R$的乘法封闭,}
    \end{align*}所以$H\cap N$是$R$的子环,命题得证.
\end{proof}
\begin{proposition}
    已知$H$是环$R$的子环,$N$是环$R$的理想,则$H\cap N$是环$H$的理想.
\end{proposition}
\begin{proof}
    注意到\begin{align*}
        \text{$H$是$R$的子环}\Lra&\text{$H$是$R$的加法子群}\\
        \text{$N$是$R$的理想}\Lra&\text{$N$是$R$的加法子群},
    \end{align*}所以由命题\ref{zqdj}可知$H\cap N$是$R$的加法子群.

    由于$N$是环$R$的理想,故$\forall n\in H\cap N,h\in H$\begin{align}
        &n\in N,h\in H\subset R\Lra nh,hn\in N\label{klm}\\
        &n\in H,h\in H\Lra nh,hn\in H,\label{sadklm}
    \end{align}由\eqref{klm}与\eqref{sadklm}便知$nh,hn\in H\cap N$,从而$H\cap N$是$H$的理想.
\end{proof}
\begin{definition}[子环的和]
    已知$H,N$是环$R$的子环,则称集合\begin{align*}
        \l\{h+n:h\in H,n\in N\r\}
    \end{align*}为子环$H,N$的和,记作$H+N$.
\end{definition}
\begin{proposition}
    子环的和不一定是子环,但子环的和关于环的加法成群.
\end{proposition}
\begin{proof}
    对于第一个论断,考虑以下反例\begin{*example}
        已知$R$是数域$\F$上全体$2$阶方阵关于矩阵的加法与乘法所成的环,令\begin{align*}
            &H=\l\{\l(\begin{matrix}
                0&0\\
                a&0
            \end{matrix}\r):a\in\F\r\}\\
            &N=\l\{\l(\begin{matrix}
                0&b\\
                0&0
            \end{matrix}\r):b\in\F\r\},
        \end{align*}容易验证$H,N$是$R$的子环,$H+N$不是$R$的子环.
    \end{*example}
    对于第二个论断,由群的第一同构定理(定理\ref{qddytgdl})即得成立.
\end{proof}
\begin{proposition}
    已知$H$是环$R$的子环,$N$是$R$的理想,则$H+N$是$R$子环.
\end{proposition}
\begin{proof}
    由群的第一同构定理(定理\ref{qddytgdl})可知,$H+N$是$R$的加法子环,所以只要证明$H+N$关于$R$的乘法封闭即可.

    $\forall x,y\in H,N,\exists x_1,y_1\in H,x_2,y_2\in N\st x=x_1+x_2,y=y_1+y_2$,所以\begin{align*}
        xy=&(x_1+x_2)(y_1+y_2)\\
        =&x_1y_1+x_1y_2+x_2y_1+x_2y_2,
    \end{align*}注意到$x_1y_2\in H,x_1y_2,x_2y_2,x_2y_2\in N$,所以\begin{align*}
        xy\in H+N,
    \end{align*}这就证明了$H+N$关于$R$的乘法封闭.

    综上,问题得证.
\end{proof}
\begin{proposition}
    已知$H,N$是环$R$的理想,则$H+N$是$R$的理想.
\end{proposition}
\begin{proof}
    由群的第一同构定理(定理\ref{qddytgdl})可知,$H+N$是$R$的加法子群,所以只要证明$H+N$关于$R$的乘法具有吸收性即可.

    注意到\begin{align*}
        &\forall a\in H+N,\exists a_1\in H,a_2\in N\st a=a_1+a_2\\
        \Lra&\forall r\in R,\\
        &ar=(a_1+a_2)r=a_1r+a_2r\in H+N\\
        &ra=r(a_1+a_2)=ra_1+ra_2\in H+N,
    \end{align*}所以$H+N$是$R$的理想.
\end{proof}
\begin{proposition}
    已知$H,N$是环$R$的子环,理想,则$N$是$H+N$的理想.
\end{proposition}
\begin{proof}
    $N$是$R$的理想\begin{align*}
        \iff&\text{$\forall r\in R,n\in N$,都有}nr\in N,rn\in N\\
        \Lra&\forall r_0\in H+N,n\in N,\text{都有}n_0r\in N,rn_0\in N,
    \end{align*}从而$N$是$H+N$的理想,命题证毕.
\end{proof}
下面开始介绍环的同态定理.
\begin{theorem}[环的第一同构定理]\label{hddytgdl}
    已知$H,N$是环$R$的子环,理想,则\begin{align*}
        H/H\cap N\cong H+N/N.
    \end{align*}
\end{theorem}
\begin{proof}
    \stars
    由群的第一同构定理(定理\ref{qddytgdl})可知有群同构\begin{align*}
        H/H\cap N\cong H+N/N,
    \end{align*}其中的同构映射为\begin{align*}
        \phi:H/H\cap N\to&H+N/N\\
        h+H\cap N\mapsto h+N,
    \end{align*}只需证明$\phi$保持乘法运算即可得到$\phi$是同构映射,从而定理成立.

    验证即可11111111
\end{proof}
\begin{theorem}\label{p102dl2}
    已知$R$是环,$\sigma:R\to\im\sigma$是同态映射,则映射\begin{align*}
        \phi:A\to&B\\
        H\mapsto&\im\sigma|_{H}
    \end{align*}是一一对应且理想与理想对应,其中\begin{align*}
        &A=\l\{H:H\supset\ker\sigma,H\text{是}R\text{的子环}\r\}\\
        &B=\l\{H_0:H_0\text{是}\im\sigma\text{的子环}\r\}.
    \end{align*}
\end{theorem}
\begin{proof}
    \stars
\end{proof}
\begin{proposition}
    已知$R$是环,$H,N$是$R$的理想且$N\subset H$.则$H/N$是$R/N$的理想.
\end{proposition}
\begin{proof}
    任取$hN\in H/N,rN\in R/N$,有\begin{align*}
        &hN\cdot rN\\
        \xLra[]{(N,+)\lhd(R,+)}&=hrN\\
        \xLra[]{\text{$H$是$R$的理想}}&\in HN=H/N,
    \end{align*}同理可证$rN\cdot hN\in H/N$,从而$H/N$是$R/N$的理想,命题得证.
\end{proof}
\begin{theorem}[环的第二同构定理]\label{hddetgdl}
    已知$R$是环,$H,N$是$R$的理想且$N\subset H$.则\begin{align*}
        (R/N)/(H/N)\cong R/H.
    \end{align*}
\end{theorem}
\begin{proof}
    \stars
    由群的第二同构定理(定理\ref{qddetgdl})可得如下群同构\begin{align*}
        (R/N)/(H/N)\cong R/H,
    \end{align*}其中的同构映射为\begin{align*}
        \phi:R/N\to&R/H\\
        r+N\mapsto&r+H,
    \end{align*}只需证明$\phi$保持乘法运算即可.

    1
\end{proof}
\begin{remark}
    取$H=\{e\}$即得环同态基本定理(定理\ref{httjbdl}).
\end{remark}
环同态基本定理(定理\ref{httjbdl}),环的第一同构定理(定理\ref{hddytgdl}),环的第二同构定理(定理\ref{hddetgdl})统称为环的同态定理.
\section{环的直和}
\begin{definition}[环的直和]\label{hdvh}
    已知$R_1\times R_2\times\cdots\times R_s$为环$R_1,\cdots,R_s$的$\mr{Cartesian}$积,对$a_k,b_k\in R_k$,定义\begin{align*}
        &(a_1,a_2,\cdots,a_s)+(b_1,b_2,\cdots,b_s)=(a_1+b_1,a_2+b_2,\cdots,a_s+b_s)\\
        &(a_1,a_2,\cdots,a_s)\cdot(b_1,b_2,\cdots,b_s)=(a_1b_1,a_2b_2,\cdots,a_sb_s),
    \end{align*}则$R_1\times R_2\times\cdots\times R_s$在新定义的加法与乘法下成环,称其为环$R_1,R_2,\cdots,R_s$的直和,记作\begin{align*}
        \bigoplus_{k=1}^sR_k.
    \end{align*}
\end{definition}
\begin{remark}
    (1)环$\l(R_1\times R_2\times\cdots\times R_s,+,\cdot\r)$的零元为$(0,0,\cdots,0)$.

    (2)若环$R_i$有幺元$1_i$,则环$\l(R_1\times\cdots\times R_s,+,\cdot\r)$有幺元$(1_1,\cdots,1_s)$.
    
    
    (3)$\bigoplus_{i=1}^{s}R_i=\sum_{i=1}^sR_i.$
    
    (4)$\bigoplus_{k=1}^sR_k$中元素可惟一地分解为$R_i$中元素的和.
    
    (5)$\forall r_i\in R_i,r_j\in R_j\text{且}i\neq j,\text{有}r_i\cdot r_j=(0,0,\cdots,0)$.

    (6)若$R_i$为交换环,则$\bigoplus_{k=1}^sR_k$为交换环.

    (7)令$R_k'=\l(\l\{0,\cdots,0,r_k,0,\cdots,0:r_k\in R_k\r\},+,\cdot\r)$,则\begin{align*}
        &(7.1)\text{$R_i'$是$\bigoplus_{k=1}^sR_k$的理想且$R_i'\cong R_i$}\\
        &(7.2)R_i'\cap\l(\sum_{j\neq i}R_j\r)=(0,0,\cdots,0).
    \end{align*}\qed
\end{remark}
\begin{theorem}\label{sdkjfh}
    若环$R$的子环$R_1,R_2,\cdots,R_s$适合\begin{align*}
        &(1)\text{$R_i$是$R$的理想}\\
        &(2)R=\sum_{i=1}^{s}R_i\\
        &(3)R_i\cap\l(\sum_{j\neq i}^{j}R_j\r)=\l(0,0,\cdots,0\r),
    \end{align*}则\begin{align*}
        R\cong\bigoplus_{i=1}^{r}R_i.
    \end{align*}
\end{theorem}
\begin{proof}
    \stars
\end{proof}
\begin{definition}[内直和]
    若环$R$的子环$R_1,R_2,\cdots,R_s$适合定理\ref{sdkjfh}的条件,则称$R$为$R_1,R_2,\cdots,R_s$的内直和.
\end{definition}
\begin{definition}[理想的积]\label{lxdj}
    设$H,N$是环$R$的理想,则有限和\begin{align*}
        \sum a_ib_i,a_i\in H,b_i\in N
    \end{align*}组成的集合是$R$的理想,称其为$H$和$N$的积,记作$H\cdot N.$\qed
\end{definition}
\begin{remark}
    理想的乘法对理想的加法满足分配律.\qed
\end{remark}
\begin{definition}[互素]\label{hs}
    若幺环$R$的理想$H,N$满足$H+N=R$,则称$H,N$互素.
\end{definition}
\begin{lemma}
    设$H,N,K$是幺环$R$的理想,则

    (1)若$R$交换,则$H,N$互素是$H\cdot N=H\cap N$的充分条件.

    (2)若$H,K$均与$N$互素,则$H\cdot K$与$N$互素.
\end{lemma}
\begin{proof}
    \stars
\end{proof}
\begin{theorem}\label{p105dl5}
    设幺环$R$的理想$N_1,\cdots,N_r$两两互素,则\begin{align*}
        R\l/\bigcap_{i=1}^{r}N_i\r.\cong\bigoplus_{i=1}^{r}R/N_i.
    \end{align*}
\end{theorem}
\begin{proof}
    \stars
\end{proof}
整数环$\l(\Z,+,\cdot\r)$的概念可以推广到任意环.
\begin{definition}[模$N$同余]
    已知$N$是环$R$的理想,对$\forall a,b\in R$,若$a-b\in N$,则称$a,b$模$N$同余,记作\begin{align*}
        a\equiv b\mod N,
    \end{align*}反之,称$a,b$模$N$非同余,记作\begin{align*}
        a\not\equiv b\mod N.
    \end{align*}
\end{definition}
\begin{theorem}[中国剩余定理]\label{vguydl}
    若幺环$R$的理想$N_1,\cdots,N_r$两两互素,则$\forall b_1,\cdots,b_r\in R$,同余方程组\begin{align*}
        \l\{\begin{array}{c}
            x\equiv b_1\mod N_1\\
            x\equiv b_2\mod N_2\\
            \cdots\\
            x\equiv b_r\mod N_r
        \end{array}\r.
    \end{align*}在$R$内恒有解,且任意两解模$\bigcap_{i=1}^{r}N_i$同余.
\end{theorem}
\begin{proof}
    \stars
\end{proof}
定理\ref{p105dl5}有一个特殊情况值得提一下,即
\begin{theorem}\label{p107dl5'}
    若幺环$R$的理想$N_1,\cdots,N_r$两两互素,$\bigcap_{i=1}^{r}N_i=\l(0,\cdots,0\r)$,则\begin{align*}
        R\cong\bigoplus_{i=1}^{r}R/N_i.
    \end{align*}
\end{theorem}
\begin{proof}
    \stars
\end{proof}
\begin{example}
    \stars
\end{example}
\begin{definition}[由$S$生成的理想,生成元集,有限生成的,主理想]\label{ysuidlx}\label{uiyj}\label{yxuid}\label{vlx}
    已知$S$是任一环$R$的非空子集.称$R$中所有包含$S$的理想的交为由$S$生成的理想,记作$(S)$,同时包含$S$的最小理想.称$S$为$(S)$的生成元集.若$S=\l\{a_1,\cdots,a_r\r\}$为有限集,则称$(S)$为有限生成的,$(S)$记作$(a_1,\cdots,a_r)$.称由一个元$a$生成的理想$(a)$为主理想.
\end{definition}
\begin{proposition}
    环$R$的主理想$(a)$是由下列形状元素\begin{align*}
        na,xa,ay,xay
    \end{align*}的一切有限和组成的,其中$n\in\Z,x,y\in R$.
\end{proposition}
\begin{proof}
    \stars
\end{proof}
\begin{proposition}
    交换环$R$的主理想$(a)$是由下列形状元素\begin{align*}
        na,xa
    \end{align*}的一切有限和组成的,其中$n\in\Z,x\in R$.
\end{proposition}
\begin{proof}
    \stars
\end{proof}
\begin{proposition}
    交换幺环$R$的主理想$(a)$是由下列形状元素\begin{align*}
        xa
    \end{align*}的一切有限和组成的,其中$x\in R$.
\end{proposition}
\begin{proof}
    \stars
\end{proof}
\section{环的反同构}
在数域$\F$上全体$n$阶矩阵关于矩阵的加法与乘法构成的环$M_n(\F)$上,称转置映射\begin{align*}
    \phi:M_n(\F)\to&M_n(\F)\\
    A\mapsto&A^T
\end{align*}为$M_n(\F)$上的反自同构,将其延伸到一般环上就有\begin{definition}[反同构,反自同构]\label{ftg}\label{fztg}
    若环$R$到$R'$的一一对应$\sigma$满足\begin{align*}
        &(1)\sigma(a+b)=\sigma(a)+\sigma(b)\\
        &(2)\sigma(a\cdot b)=\sigma(b)\cdot\sigma(a),
    \end{align*}则称$\sigma$为$R$到$R'$的反自同构,也称$R$与$R'$反同构.当$R'=R$时,称$\sigma$为$R$的反自同构.
\end{definition}
\begin{proposition}
    若环$R$与环$R'$成反同构且$R$交换,则$R'$也交换,反同构也是同构.
\end{proposition}
\begin{proof}
    \stars
\end{proof}
\begin{proposition}\label{sdkf}
    环$R$的自同构与反自同构关于映射的乘法成群.
\end{proposition}
\begin{proof}
    \stars
\end{proof}
\begin{proposition}
    记命题\ref{sdkf}中的群为$G$.则对于命题\ref{sdkf}中的$R$有\begin{align*}
        &(1)\Aut(R)<G.\\
        &(2)[G:\Aut(R)]\leq2.
    \end{align*}
\end{proposition}
\begin{proof}
    \stars
\end{proof}
\begin{proposition}
    对任一非交换环$R$,可作出环$R'$使得$R'$与$R$成反同构.
\end{proposition}
\begin{proof}
    \stars
\end{proof}
\begin{example}
    存在没有反自同构的交换环.
\end{example}
\begin{proof}
    \stars
\end{proof}