第一章介绍了环,理想,商环以及环同态等概念并建立了环同态基本定理.本章将进一步给出环的几个重要的同态定理与几种构造环的方法.
\section{环的同态定理}
在给出环的同态定理之前,先介绍理想的运算.
\begin{proposition}
    子群的交仍是子群.

    \stars
\end{proposition}
\begin{proof}
    1%\stars
\end{proof}
\begin{proposition}\label{jkasdnf}
    已知$H,N$是环$R$的子环,则$H\cap N$是环$R$的子环.
\end{proposition}
\begin{proof}
    由于$H\cap N$是$R$的加法子Abel群,所以只需证明$H\cap N$是关于$R$的乘法封闭即可.
    
    $\forall x,y\in H\cap N$注意到\begin{align*}
        x,y\in H\Lra&xy\in H\\
        x,y\in N\Lra&xy\in N,
    \end{align*}可见\begin{align*}
        \forall x,y\in H\cap N\Lra&xy\in H\cap N\\
        \iff&\text{$H\cap N$关于$R$的乘法封闭,}
    \end{align*}所以$H\cap N$是$R$的子环,命题得证.
\end{proof}
\begin{proposition}
    已知$H$是环$R$的子环,$N$是环$R$的理想,则$H\cap N$是环$H$的理想.
\end{proposition}
\begin{proof}
    注意到\begin{align*}
        \text{$H$是$R$的子环}\Lra&\text{$H$是$R$的加法子群}\\
        \text{$N$是$R$的理想}\Lra&\text{$N$是$R$的加法子群},
    \end{align*}所以$H\cap N$是$R$的加法子群.

    由于$N$是环$R$的理想,故$\forall n\in H\cap N,h\in H$\begin{align}
        &n\in N,h\in H\subset R\Lra nh,hn\in N\label{klm}\\
        &n\in H,h\in H\Lra nh,hn\in H,\label{sadklm}
    \end{align}由\eqref{klm}与\eqref{sadklm}便知$nh,hn\in H\cap N$,从而$H\cap N$是$H$的理想.
\end{proof}
\begin{definition}[子环的和]
    已知$H,N$是环$R$的子环,则称集合\begin{align*}
        \l\{h+n:h\in H,n\in N\r\}
    \end{align*}为子环$H,N$的和,记作$H+N$.
\end{definition}
\begin{remark}
    子环的和不一定是子环,但子环的和关于环的加法成群.
\end{remark}
\begin{proof}
    对于第一个论断,考虑以下反例\begin{*example}
        已知$R$是数域$\F$上全体$2$阶方阵关于矩阵的加法与乘法所成的环,令\begin{align*}
            &H=\l\{\l(\begin{matrix}
                0&0\\
                a&0
            \end{matrix}\r):a\in\F\r\}\\
            &N=\l\{\l(\begin{matrix}
                0&b\\
                0&0
            \end{matrix}\r):b\in\F\r\},
        \end{align*}容易验证$H,N$是$R$的子环,$H+N$不是$R$的子环.
    \end{*example}
    对于第二个论断,由群的第一同构定理(定理\ref{qddytgdl})即得成立.
\end{proof}
\begin{proposition}
    已知$H$是环$R$的子环,$N$是$R$的理想,则$H+N$是$R$子环.
\end{proposition}
\begin{proof}
    由群的第一同构定理(定理\ref{qddytgdl})可知,$H+N$是$R$的加法子环,所以只要证明$H+N$关于$R$的乘法封闭即可.

    $\forall x,y\in H,N,\exists x_1,y_1\in H,x_2,y_2\in N\st x=x_1+x_2,y=y_1+y_2$,所以\begin{align*}
        xy=&(x_1+x_2)(y_1+y_2)\\
        =&x_1y_1+x_1y_2+x_2y_1+x_2y_2,
    \end{align*}注意到$x_1y_2\in H,x_1y_2,x_2y_2,x_2y_2\in N$,所以\begin{align*}
        xy\in H+N,
    \end{align*}这就证明了$H+N$关于$R$的乘法封闭.

    综上,问题得证.
\end{proof}
\begin{proposition}
    已知$H,N$是环$R$的理想,则$H+N$是$R$的理想.
\end{proposition}
\begin{proof}
    由群的第一同构定理(定理\ref{qddytgdl})可知,$H+N$是$R$的加法子群,所以只要证明$H+N$关于$R$的乘法具有吸收性即可.

    注意到\begin{align*}
        &\forall a\in H+N,\exists a_1\in H,a_2\in N\st a=a_1+a_2\\
        \Lra&\forall r\in R,\\
        &ar=(a_1+a_2)r=a_1r+a_2r\in H+N\\
        &ra=r(a_1+a_2)=ra_1+ra_2\in H+N,
    \end{align*}所以$H+N$是$R$的理想.
\end{proof}
下面开始介绍环的同态定理.
\begin{theorem}[环的第一同构定理]\label{hddytgdl}
    已知$H,N$是环$R$的子环,理想,则\begin{align*}
        H/H\cap N\cong H+N/N.
    \end{align*}
\end{theorem}
\begin{proof}
    \stars
\end{proof}
\begin{theorem}\label{p102dl2}
    已知$R$是环,$\sigma:R\to\im\sigma$是同态映射,则映射\begin{align*}
        \phi:A\to&B\\
        H\mapsto&\im\sigma|_{H}
    \end{align*}是一一对应且理想与理想对应,其中\begin{align*}
        &A=\l\{H:H\supset\ker\sigma,H\text{是}R\text{的子环}\r\}\\
        &B=\l\{H_0:H_0\text{是}\im\sigma\text{的子环}\r\}.
    \end{align*}
\end{theorem}
\begin{proof}
    \stars
\end{proof}
\begin{theorem}[环的第二同构定理]\label{hddetgdl}
    已知$R$是环,$\sigma:R\to\im\sigma$是同态映射,$H$是$R$的理想且$H\supset\ker\sigma$,则\begin{align*}
        R/H\cong\l(R/\ker\sigma\r)/\l(H/\ker\sigma\r).
    \end{align*}
\end{theorem}
\begin{proof}
    \stars
\end{proof}
\begin{remark}
    取$H=\ker\sigma$即得环同态基本定理(定理\ref{httjbdl}).
\end{remark}
将环的第一同构定理(定理\ref{hddytgdl})与环的第二同构定理(定理\ref{hddetgdl})统称为环的同态定理.